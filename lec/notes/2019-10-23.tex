\documentclass[12pt, leqno]{article} %% use to set typesize
\usepackage{fancyhdr}
\usepackage[sort&compress]{natbib}
\usepackage[letterpaper=true,colorlinks=true,linkcolor=black]{hyperref}

\usepackage{amsfonts}
\usepackage{amsmath}
\usepackage{amssymb}
\usepackage{color}
\usepackage{tikz}
\usepackage{pgfplots}
\usepackage{listings}
\usepackage{courier}
%\usepackage[utf8]{inputenc}
%\usepackage[russian]{babel}

\lstset{
  numbers=left,
  basicstyle=\ttfamily\footnotesize,
  numberstyle=\tiny\color{gray},
  stepnumber=1,
  numbersep=10pt,
}

\newcommand{\iu}{\ensuremath{\mathrm{i}}}
\newcommand{\bbR}{\mathbb{R}}
\newcommand{\bbC}{\mathbb{C}}
\newcommand{\calV}{\mathcal{V}}
\newcommand{\calW}{\mathcal{W}}
\newcommand{\macheps}{\epsilon_{\mathrm{mach}}}
\newcommand{\matlab}{\textsc{Matlab}}

\newcommand{\ddiag}{\operatorname{diag}}
\newcommand{\fl}{\operatorname{fl}}
\newcommand{\nnz}{\operatorname{nnz}}
\newcommand{\tr}{\operatorname{tr}}
\renewcommand{\vec}{\operatorname{vec}}

\newcommand{\vertiii}[1]{{\left\vert\kern-0.25ex\left\vert\kern-0.25ex\left\vert #1
    \right\vert\kern-0.25ex\right\vert\kern-0.25ex\right\vert}}
\newcommand{\ip}[2]{\langle #1, #2 \rangle}
\newcommand{\ipx}[2]{\left\langle #1, #2 \right\rangle}
\newcommand{\order}[1]{O( #1 )}

\newcommand{\kron}{\otimes}


\newcommand{\hdr}[2]{
  \pagestyle{fancy}
  \lhead{Bindel, Fall 2019}
  \rhead{Matrix Computation}
  \fancyfoot{}
  \begin{center}
    {\large{\bf HW for #1}} \\ (due: #2)
  \end{center}
  \lstset{language=matlab,columns=flexible}
}


\newcommand{\calB}{\mathcal{B}}
\newtheorem{theorem}{Theorem}

\begin{document}

\hdr{2019-10-23}

\section{Similarity transforms}

When we talked about least squares problems, we spent some time
discussing the transformations that preserve the Euclidean norm:
orthogonal transformations.  It is worth spending a moment now to give
a name to the transformations that preserve the eigenvalue structure
of a matrix.  These are {\em similarity} transformations.

Suppose $A \in \bbC^{n \times n}$ is a square matrix, and $X \in
\bbC^{n \times n}$ is invertible.  Then the matrix $X A X^{-1}$ is
said to be {\em similar} to $A$, and the mapping from $A$ to $X A
X^{-1}$ is a {\em similarity transformation}.  If $A$ is the matrix for
an operator from $\bbC^n$ onto itself, then $X A X^{-1}$ is the matrix
for the same operator in a different basis.  The eigenvalues and the
Jordan block structure of a matrix are preserved under similarity, and
the matrix $X$ gives a relationship between the eigenvectors of $A$
and those of $X A X^{-1}$.  Note that this goes both ways: two
matrices have the same eigenvalues and Jordan block structure iff they
are similar.

\section{Eigenvalue perturbations: a 2-by-2 illustration}

Consider the matrix
\[
  A(\epsilon) =
  \begin{bmatrix}
     \lambda & 1 \\
    \epsilon & \lambda
  \end{bmatrix}.
\]
The characteristic polynomial of $A(\epsilon)$ is $p(z) = z^2 -
2\lambda z + (\lambda^2-\epsilon)$, which has roots $\lambda \pm
\sqrt{\epsilon}$.  These eigenvalues are {\em continuous} functions of
$\epsilon$ at $\epsilon = 0$, but they are not differentiable
functions.  This is a more general phenomenon: an $O(\epsilon)$
perturbation to a matrix with an eigenvalue with multiplicity $m$
usually splits the eigenvalue into $m$ distinct eigenvalues, each of
which is moved from the original position by $O(\epsilon^{1/m})$.  We
expect, then, that it will be difficult to accurately compute multiple
eigenvalues of general nonsymmetric matrices in floating point.  If we
are properly suspicious, we should suspect that {\em nearly} multiple
eigenvalues are almost as troublesome --- and indeed they are.  On the
other hand, while we usually lose some accuracy when trying to compute
nearly multiple eigenvalues, we should not always expect to lose
{\em all} digits of accuracy.

The next lecture or two will be spent developing the perturbation
theory we will need in order to figure out what we can and cannot
expect from our eigenvalue computations.

\section{First-order perturbation theory}

Suppose $A \in \bbC^{n \times n}$ has a simple\footnote{
  An eigenvalue is simple if it is not multiple.
} eigenvalue $\lambda$ with corresponding column
eigenvector $v$ and row eigenvector $w^*$.
We would like to understand how $\lambda$ changes under
small perturbations to $A$.  If we formally differentiate
the eigenvalue equation $A v = v \lambda$, we have
\[
  (\delta A) v + A (\delta v) = (\delta v) \lambda + v (\delta \lambda).
\]
If we multiply this equation by $w^*$, we have
\[
  w^* (\delta A) v + w^* A (\delta v) =
  \lambda w^* (\delta v) + w^* v (\delta \lambda).
\]
Note that $w^* A = \lambda w^*$, so that we have
\[
  w^* (\delta A) v = w^* v (\delta \lambda),
\]
which we rearrange to get
\begin{equation} \label{basic-sensitivity}
  \delta \lambda = \frac{w^* (\delta A) v}{w^* v}.
\end{equation}
This formal derivation of the first-order sensitivity of an
eigenvalue only goes awry if $w^* v = 0$, which we can show is
not possible if $\lambda$ is simple.

We can use formula (\ref{basic-sensitivity}) to get a condition
number for the eigenvalue $\lambda$ as follows:
\[
  \frac{|\delta \lambda|}{|\lambda|}
   = \frac{|w^* (\delta A) v|}{|w^* v| |\lambda|}
    \leq \frac{\|w\|_2 \|v\|_2}{|w^* v|} \frac{\|\delta A\|_2}{|\lambda|}
    = \sec \theta \frac{\|\delta A\|_2}{|\lambda|}.
\]
where $\theta$ is the acute angle between the spaces spanned by $v$ and by $w$.
When this angle is large, very small perturbations can drastically change the
eigenvalue.

\section{Gershgorin theory}

The first-order perturbation theory outlined in the previous section
is very useful, but it is also useful to consider the effects of
{\em finite} (rather than infinitesimal) perturbations to $A$.  One of
our main tools in this consideration will be Gershgorin's theorem.

Here is the idea.  We know that diagonally dominant matrices are nonsingular,
so if $A - \lambda I$ is diagonally dominant, then $\lambda$ cannot be an
eigenvalue.  Contraposing this statement, $\lambda$ can be an
eigenvalue only if $A - \lambda I$ is {\em not} diagonally dominant.
The set of points where $A - \lambda I$ is not diagonally dominant is
a union of sets $\cup_j G_j$, where each $G_j$ is a {\em Gershgorin disk}:
\[
  G_j = B_{\rho_j}(a_{jj}) =
  \left\{
    z \in \bbC : |a_{jj}-z| \leq \rho_j \mbox{ where }
    \rho_j = \sum_{i \neq j} |a_{ij}|
  \right\}.
\]
Our strategy now, which we will pursue in detail next time, is to use
similarity transforms based on $A$ to make a perturbed matrix $A+E$
look ``almost'' diagonal, and then use Gershgorin theory to turn that
``almost'' diagonality into bounds on where the eigenvalues can be.

We now argue that we can extract even more information from the
Gershgorin disks: we can get {\em counts} of how many eigenvalues
are in different parts of the union of Gershgorin disks.

Suppose that $\mathcal{G}$ is a connected component of $\cup_j G_j$;
in other words, suppose that $\mathcal{G} = \cup_{j \in S} G_j$ for
some set of indices $S$, and that $\mathcal{G} \cap G_k = \emptyset$
for $k \not \in S$.  Then the number of eigenvalues of $A$ in
$\mathcal{G}$ (counting eigenvalues according to multiplicity) is the
same as the side of the index set $S$.

To sketch the proof, we need to know that eigenvalues are continuous
functions of the matrix entries.  Now, for $s \in [0,1]$, define
\[
  H(s) = D + sF
\]
where $D$ is the diagonal part of $A$ and $F = A-D$ is the off-diagonal
part.  The function $H(s)$ is a {\em homotopy} that continuously takes
us from an easy-to-analyze diagonal matrix at $H(0) = D$ to the matrix
we care about at $H(1) = A$.  At $s = 0$, we know the eigenvalues of $A$
are the diagonal elements of $A$; and if we apply the first part of Gershgorin's
theorem, we see that the eigenvalues of $H(s)$ always must live inside
the union of Gershgorin disks of $A$ for any $0 \leq s \leq 1$.
So each of the $|S|$ eigenvalues that start off in the connected component
$\calG$ at $H(0) = D$ can move around continuously within $\calG$
as we move the matrix continuously to $H(1) = A$, but they cannot ``jump''
discontinuously across the gap between $\calG$ and any of the other Gershgorin
disks.  So at $s = 1$, there will still be $|S|$ eigenvalues of $H(1) = A$
inside $\calG$.

\section{Perturbing Gershgorin}

Now, let us consider the relation between the Gershgorin disks for
a matrix $A$ and a matrix $\hat{A} = A+F$.  It is straightforward to write
down the Gershgorin disks $\hat{G}_j$ for $\hat{A}$:
\[
  \hat{G}_j = \calB_{\hat{\rho}_j}(\hat{a}_{jj}) =
  \left\{
    z \in \bbC : |a_{jj}+e_{jj}-z| \leq \hat{\rho}_j
  \right\}
    \mbox{ where }
    \hat{\rho}_j = \sum_{i \neq j} |a_{ij}+f_{ij}|.
\]
Note that $|a_{jj} + e_{jj} - z| \geq |a_{jj}-z|-|f_{jj}|$
and $|a_{ij}+f_{ij}| \leq |a_{ij}|+|f_{ij}|$, so
\begin{equation} \label{disk-bound-1}
  \hat{G}_j \subseteq \calB_{\rho_j + \sum_j |f_{ij}|}(a_{jj}) =
  \left\{
    z \in \bbC : |a_{jj}-z| \leq \rho_j + \sum_i |f_{ij}|
  \right\}.
\end{equation}
We can simplify this expression even further if we are willing
to expand the regions a bit:
\begin{equation} \label{disk-bound-2}
  \hat{G}_j \subseteq \calB_{\rho_j + \|F\|_{1}}(a_{jj}).
\end{equation}

\section{The Bauer-Fike theorem}

We now apply Gershgorin theory together with a carefully chosen
similarity to prove a bound on the eigenvalues of $A+F$ where $F$ is a
finite perturbation.  This will lead us to the {\em Bauer-Fike} theorem.

The basic idea is as follows.  Suppose that $A$ is a diagonalizable matrix, so
that there is a complete basis of column eigenvectors $V$ such that
\[
  V^{-1} A V = \Lambda.
\]
Then we $A+F$ has the same eigenvalues as
\[
  V^{-1} (A+F) V = \Lambda + V^{-1} F V = \Lambda + \tilde{F}.
\]
Now, consider the Gershgorin disks for $\Lambda + \tilde{F}$.
The crude bound (\ref{disk-bound-2}) tells us
that all the eigenvalues live in the regions
\[
  \bigcup_j \calB_{\|\tilde{F}\|_1}(\lambda_j) \; \subseteq \;
  \bigcup_j \calB_{\kappa_1(V) \|F\|_1}(\lambda_j).
\]
This bound really is crude, though; it gives us disks of the
same radius around all the eigenvalues $\lambda_j$ of $A$,
regardless of the conditioning of those eigenvalues.  Let's
see if we can do better with the sharper bound (\ref{disk-bound-1}).

To use (\ref{disk-bound-1}), we need to bound
the absolute column sums of $\tilde{F}$.  Let $e$ represent
the vector of all ones, and let $e_j$ be the $j$th column of
the identity matrix; then the $j$th absolute column sums of $\tilde{F}$
is $\phi_j \equiv e^T |\tilde{F}| e_j$, which we can bound
as $\phi_j \leq e^T |V^{-1}| |F| |V| e_j$.  Now, note that we
are free to choose the normalization of the eigenvector $V$;
let us choose the normalization so that each row of $W^* = V^{-1}$.
Recall that we defined the angle $\theta_j$ by
\[
  \cos(\theta_j) = \frac{|w_j^* v_j|}{\|w_j\|_2 \|v_j\|_2},
\]
where $w_j$ and $v_j$ are the $j$th row and column eigenvectors;
so if we choose $\|w_j\|_2 = 1$ and $w_j^* v_j = 1$ (so $W^* = V^{-1}$),
we must have $\|v_j\|_2 = \sec(\theta_j)$.  Therefore,
$\||V| e_j\|_2 = \sec(\theta_j)$.  Now, note that $e^T |V^{-1}|$ is
a sum of $n$ rows of Euclidean length 1, so $\|e^T |V^{-1}|\|_2 \leq n$.
Thus, we have
\[
  \phi_j \leq n \|F\|_2 \sec(\theta_j).
\]
Putting this bound on the columns of $\tilde{F}$ together with
(\ref{disk-bound-1}), we have the Bauer-Fike theorem.

\begin{theorem}
  Suppose $A \in \bbC^{n \times n}$ is diagonalizable with
  eigenvalues $\lambda_1, \ldots, \lambda_n$.
  Then all the eigenvalues of $A+F$ are in the region
  \[
    \bigcup_j \calB_{n \|F\|_2 \sec(\theta_j)}(\lambda_j),
  \]
  where $\theta_j$ is the acute angle between the row and column eigenvectors
  for $\lambda_j$, and any connected component $\calG$ of this region that
  contains exactly $m$ eigenvalues of $A$ will also contain exactly $m$
  eigenvalues of $A+F$.
\end{theorem}

\end{document}

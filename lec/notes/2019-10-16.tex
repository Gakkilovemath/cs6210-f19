\documentclass[12pt, leqno]{article} %% use to set typesize
\usepackage{fancyhdr}
\usepackage[sort&compress]{natbib}
\usepackage[letterpaper=true,colorlinks=true,linkcolor=black]{hyperref}

\usepackage{amsfonts}
\usepackage{amsmath}
\usepackage{amssymb}
\usepackage{color}
\usepackage{tikz}
\usepackage{pgfplots}
\usepackage{listings}
\usepackage{courier}
%\usepackage[utf8]{inputenc}
%\usepackage[russian]{babel}

\lstset{
  numbers=left,
  basicstyle=\ttfamily\footnotesize,
  numberstyle=\tiny\color{gray},
  stepnumber=1,
  numbersep=10pt,
}

\newcommand{\iu}{\ensuremath{\mathrm{i}}}
\newcommand{\bbR}{\mathbb{R}}
\newcommand{\bbC}{\mathbb{C}}
\newcommand{\calV}{\mathcal{V}}
\newcommand{\calW}{\mathcal{W}}
\newcommand{\macheps}{\epsilon_{\mathrm{mach}}}
\newcommand{\matlab}{\textsc{Matlab}}

\newcommand{\ddiag}{\operatorname{diag}}
\newcommand{\fl}{\operatorname{fl}}
\newcommand{\nnz}{\operatorname{nnz}}
\newcommand{\tr}{\operatorname{tr}}
\renewcommand{\vec}{\operatorname{vec}}

\newcommand{\vertiii}[1]{{\left\vert\kern-0.25ex\left\vert\kern-0.25ex\left\vert #1
    \right\vert\kern-0.25ex\right\vert\kern-0.25ex\right\vert}}
\newcommand{\ip}[2]{\langle #1, #2 \rangle}
\newcommand{\ipx}[2]{\left\langle #1, #2 \right\rangle}
\newcommand{\order}[1]{O( #1 )}

\newcommand{\kron}{\otimes}


\newcommand{\hdr}[2]{
  \pagestyle{fancy}
  \lhead{Bindel, Fall 2019}
  \rhead{Matrix Computation}
  \fancyfoot{}
  \begin{center}
    {\large{\bf HW for #1}} \\ (due: #2)
  \end{center}
  \lstset{language=matlab,columns=flexible}
}


\begin{document}

\hdr{2019-10-16}

\section{Factor selection and pivoted QR}

In ill-conditioned problems, the columns of $A$ are nearly linearly
dependent; we can effectively predict some columns as linear
combinations of other columns.  The goal of the column pivoted QR
algorithm is to find a set of columns that are ``as linearly
independent as possible.''  This is not such a simple task,
and so we settle for a greedy strategy: at each step, we select the
column that is least well predicted (in the sense of residual norm)
by columns already selected.  This leads to the {\em pivoted QR
  factorization}
\[
  A \Pi = Q R
\]
where $\Pi$ is a permutation and the diagonal entries of $R$ appear
in descending order (i.e. $r_{11} \geq r_{22} \geq \ldots$).  To
decide on how many factors to keep in the factorization, we either
automatically take the first $k$ or we dynamically choose to take $k$
factors where $r_{kk}$ is greater than some tolerance and
$r_{k+1,k+1}$ is not.

The pivoted QR approach has a few advantages.  It yields {\em
  parsimonious} models that predict from a subset of the columns of
$A$ -- that is, we need to measure fewer than $n$ factors to produce
an entry of $b$ in a new column.  It can also be computed relatively
cheaply, even for large matrices that may be sparse.
However, pivoted QR is not the only approach!  A related approach
due to Golub, Klema, and Stewart computes $A = U \Sigma V^T$ and
chooses a subset of the factors based on pivoted QR of $V^T$.
More generally, approaches such as the lasso yield an automatic
factor selection.

\section{Tikhonov regularization (ridge regression)}

Another approach is to say that we want a model in which the
coefficients are not too large.  To accomplish this, we add
a penalty term to the usual least squares problem:
\[
  \mbox{minimize } \|Ax-b\|^2 + \lambda^2 \|x\|^2.
\]
Equivalently, we can write
\[
\mbox{minimize } \left\|
\begin{bmatrix} A \\ \lambda I \end{bmatrix} x -
\begin{bmatrix} b \\ 0 \end{bmatrix}
\right\|^2,
\]
which leads to the regularized version of the normal equations
\[
  (A^T A + \lambda^2 I) x = A^T b.
\]
In some cases, we may want to regularize with a more general
norm $\|x\|_M^2 = x^T M x$ where $M$ is symmetric and positive
definite, which leads to the regularized equations
\[
  (A^T A + \lambda^2 M) x = A^T b.
\]
If we want to incorporate prior information that pushes $x$
toward some initial guess $x_0$, we may pose the least squares
problem in terms of $z = x-x_0$ and use some form of Tikhonov
regularization.  If we know of no particular problem structure
in advance, the standard choice of $M = I$ is a good default.

It is useful to compare the usual least squares solution to the
regularized solution via the SVD.  If $A = U \Sigma V^T$ is the
economy SVD, then
\begin{align*}
  x_{LS} &= V \Sigma^{-1} U^T b \\
  x_{Tik} &= V f(\Sigma)^{-1} U^T b
\end{align*}
where
\[
  f(\sigma) = \frac{1}{\sqrt{\sigma^{-1} + \lambda^2}}.
\]
This {\em filter} of the inverse singular values affects the larger
singular values only slightly, but damps the effect of very small
singular values.

\section{Truncated SVD}

The Tikhonov filter reduces the effect of small singular values on
the solution, but it does not eliminate that effect.  By contrast,
the {\em truncated SVD} approach uses the filter
\[
f(z) =
\begin{cases}
  z, & z > \sigma_{\min} \\
  \infty, & \mbox{otherwise}.
\end{cases}
\]
In other words, in the truncated SVD approach, we use
\[
  x = V_k \Sigma_k^{-1} U_k^T b
\]
where $U_k$ and $V_k$ represent the leading $k$ columns of $U$ and
$V$, respectively, while $\Sigma_k$ is the diagonal matrix consisting
of the $k$ largest singular values.

\section{$\ell^1$ and the lasso}

An alternative to Tikhonov regularization (based on a Euclidean norm
of the coefficient vector) is an $\ell^1$ regularized problem
\[
  \mbox{minimize } \|Ax-b\|^2 + \lambda \|x\|_1.
\]
This is sometimes known as the ``lasso'' approach.  The $\ell^1$
regularized problem has the property that the solutions tend to
become sparse as $\lambda$ becomes larger.  That is, the $\ell^1$
regularization effectively imposes a factor selection process like
that we saw in the pivoted QR approach.  Unlike the pivoted QR
approach, however, the $\ell^1$ regularized solution cannot be
computed by one of the standard factorizations of numerical linear
algebra.  Instead, one treats it as a more general {\em convex
  optimization} problem.

\section{Regularization via iteration}

We have briefly talked about one iterative method already (iterative
refinement), and will talk about other iterative methods later in the
semester.  Some of these iterations have a regularizing effect when
they are truncated early.  In fact, there is an argument that slowly
convergent methods may be beneficial in some cases!

As an example, consider the {\em Landweber iteration}, which is
gradient descent applied to linear least squares problems:
\[
  x^{k+1} = x^k - \alpha_k A^T (Ax^k-b).
\]
If we start from the initial guess $x^0 = 0$ and let the step size be
a fixed $\alpha_k = \alpha$, each subsequent step is a partial sum of
a Neumann series
\begin{align*}
  x^{k+1}
  &= \sum_{j=0}^k (I-\alpha A^T A)^j \alpha A^T b \\
  &= \left(I - (I-\alpha A^TA)^{k+1} \right) (\alpha A^TA)^{-1} \alpha A^T b \\
  &= \left(I - (I-\alpha A^TA)^{k+1} \right) A^\dagger b.
\end{align*}
Alternately, we can write the iterates in terms of the singular value
decomposition with a filter for regularization:
\[
  x^{k+1} = V \tilde{\Sigma}^{-1} U^T b, \quad
  \tilde{\sigma}_j^{-1} = (1-(1-\alpha \sigma_j^2)^{k+1}) \sigma_j^{-1}.
\]
Usually, the Landweber iteration is stopped when $k$ is large enough
so that the filter is nearly the identity for large singular values,
but is small enough to suppress the influence of small singular
values.

The Landweber iteration is not alone in having a regularizing effect,
but it is easier to analyze than some of the more sophisticated
Krylov-based methods that we will describe later in the semester.

\section{Tradeoffs and tactics}

All five of the regularization approaches we have described are used
in practice, and each has something to recommend it.  The pivoted QR
approach is relatively inexpensive, and it results in a model that
depends on only a few factors.  If taking the measurements to compute
a prediction costs money --- or even costs storage or bandwidth for
the factor data! --- such a model may be to our advantage.  The
Tikhonov approach is likewise inexpensive, and has a nice Bayesian
interpretation (though we didn't talk about it).  The truncated SVD
approach involves the best approximation rank $k$ approximation to the
original factor matrix, and can be interpreted as finding the $k$ best
factors that are linear combinations of the original measurements.
The $\ell_1$ approach again produces models with sparse coefficients;
but unlike QR with column pivoting, the $\ell_1$ regularized solutions
incorporate information about the vector $b$ along with the matrix $A$.
Regularization via iteration may be particularly appropriate for
large-scale problems.

So which regularization approach should one use?  In terms of
prediction quality, all can provide a reasonable deterrent against
ill-posedness and overfitting due to highly correlated factors.  Also,
all of the methods described have a parameter (the number of retained
factors, or a penalty parameter $\lambda$) that governs the tradeoff
between how well-conditioned the fitting problem will be and the
increase in bias that naturally comes from looking at a smaller class
of models.  Choosing this tradeoff intelligently may be rather more
important than the specific choice of regularization strategy.

\section{Choice of regularization}

All of the regularization methods we have discussed share a common
trait: they define a parametric family of models.  With more
regularization, we restrict the range of models we can easily generate
(adding bias), but we also reduce the sensitivity of the fit (reducing
variance). The choice of the regularization parameter is a key aspect of
these methods, and we now briefly discuss three different ways of
systematically making that choice.  In all cases, we
rely on the assumption that the sample observations we use for the
fit are representative of the population of observations where we might
want to predict.

\subsection{Morozov's discrepancy principle}

Suppose that we want to fit $Ax \approx \hat{b}$ by regularized least
squares, and the (noisy) observation vector $\hat{b}$ is known to be
within some error bound $\|e\|$ of the true values $b$. The discrepancy
principle says that we should choose the regularization parameter so the
residual norm is approximately $\|e\|$. That is, we seek the most stable
fitting problem we can get subject to the constraint that the residual
error for the regularized solution (with the noisy vector $\hat{b}$) is
not much bigger than we would get from unknown true solution.

One of the most obvious drawbacks of the discrepancy principle is that
it requires that we have an estimate for the norm of the error in the
data.  Sadly, such estimates are not always available.

\subsection{The L-curve}

A second approach to the regularization parameter is the {\em L-curve}.
If we draw a parametric curve of the residual error versus solution norm
on a log-log plot, with $\log \|r_{\lambda}\|$ on the $x$ axis
and $\log \|x_{\lambda}\|$ on the $y$ axis, we often see an ``L'' shape.
In the top of the vertical bar (small $\lambda$), we find that increasing
regularization decreases the solution norm significantly without significantly
increasing the residual error.  Along the end of the horizontal part,
increasing regularization increases the residual error, but does not
significantly help with the solution norm.  We want the corner of the
curve, where the regularization is chosen to minimize the norm of the
solution subject to the constraint that the residual is close to the
smallest possible residual (which we would have without regularization).

Computing the inflection point on the L-curve is a neat calculus
exercise which we will not attempt here.

\subsection{Cross-validation}

The idea with cross-validation is to choose the parameter
by fitting the model on a subset of the data and testing on the remaining
data.  We may do this with multiple partitions into data used for fitting
versus data reserved for checking predictions.
We often choose regularization parameters to give the smallest error on
the predictions in a cross-validation study.

One variant of cross-validation involves minimizing the
{\em leave-one-out cross-validation} (LOOCV) statistic:
\[
  \mbox{LOOCV} = \frac{1}{m} \sum_{i=1}^m \left[ Ax^{(-i)}-b \right]_i^2,
\]
where $x^{(-i)}$ denotes the model coefficients fit using all but the
$i$th data point.

To compute the LOOCV statistic in the most obvious way, we would
delete each row $a_i^T$ of $A$ in turn, fit the model coefficients
$x^{(-i)}$, and then evaluate $r^{(-i)} = b_i - a_i^T x^{(-i)}$.
This involves $m$
least squares problems, for a total cost of $O(m^2 n^2)$ (as opposed
to the usual $O(mn^2)$ cost for an ordinary least squares problem).
Let us find a better way!

The key is to write the equations for $x^{(-i)}$ as a small change to
the equations for $A^T A x^* = A^T b$:
\[
  (A^T A - a_i a_i^T) x^{(-i)} = A^T b - a_i b_i.
\]
This subtracts the influence of row $i$ from both sides of the normal
equations.  By introducing the auxiliary variable $\gamma = -a_i^T x^{(-i)}$,
we have
\[
  \begin{bmatrix}
    A^TA & a_i \\
    a_i^T & 1
  \end{bmatrix}
  \begin{bmatrix} x^{(-i)} \\ \gamma \end{bmatrix} =
  \begin{bmatrix} A^T b - a_i b_i \\ 0 \end{bmatrix}.
\]
Eliminating $x^{(-i)}$ gives
\[
  (1-\ell_i^2) \gamma = \ell_i^2 b_i - a_i^T x^*
\]
where $\ell_i^2 = a_i^T (A^T A)^{-1} a_i$ is called the
{\em leverage score} for row $i$.  Now, observe that
if $r = b-Ax^*$ is the residual for the full problem, then
\[
(1-\ell_i^2) r^{(-i)}
  = (1-\ell_i^2) (b_i + \gamma)
  = (1-\ell_i^2) b_i + \ell_i^2 b_i - a_i^T x_*
  = r_i,
\]
or, equivalently
\[
  r^{(-i)} = \frac{r_i}{1-\ell_i^2}.
\]
We finish the job by observing that $\ell_i^2$ is the $i$th diagonal
element of the orthogonal projector $\Pi = A(A^TA)A^{-1}$, which we
can also write in terms of the economy QR decomposition of $A$ as
$\Pi = QQ^T$.  Hence, $\ell_i^2$ is the squared row sum of $Q$ in
the QR factorization.


\end{document}

\documentclass[12pt, leqno]{article} %% use to set typesize
\usepackage{fancyhdr}
\usepackage[sort&compress]{natbib}
\usepackage[letterpaper=true,colorlinks=true,linkcolor=black]{hyperref}

\usepackage{amsfonts}
\usepackage{amsmath}
\usepackage{amssymb}
\usepackage{color}
\usepackage{tikz}
\usepackage{pgfplots}
\usepackage{listings}
\usepackage{courier}
%\usepackage[utf8]{inputenc}
%\usepackage[russian]{babel}

\lstset{
  numbers=left,
  basicstyle=\ttfamily\footnotesize,
  numberstyle=\tiny\color{gray},
  stepnumber=1,
  numbersep=10pt,
}

\newcommand{\iu}{\ensuremath{\mathrm{i}}}
\newcommand{\bbR}{\mathbb{R}}
\newcommand{\bbC}{\mathbb{C}}
\newcommand{\calV}{\mathcal{V}}
\newcommand{\calW}{\mathcal{W}}
\newcommand{\macheps}{\epsilon_{\mathrm{mach}}}
\newcommand{\matlab}{\textsc{Matlab}}

\newcommand{\ddiag}{\operatorname{diag}}
\newcommand{\fl}{\operatorname{fl}}
\newcommand{\nnz}{\operatorname{nnz}}
\newcommand{\tr}{\operatorname{tr}}
\renewcommand{\vec}{\operatorname{vec}}

\newcommand{\vertiii}[1]{{\left\vert\kern-0.25ex\left\vert\kern-0.25ex\left\vert #1
    \right\vert\kern-0.25ex\right\vert\kern-0.25ex\right\vert}}
\newcommand{\ip}[2]{\langle #1, #2 \rangle}
\newcommand{\ipx}[2]{\left\langle #1, #2 \right\rangle}
\newcommand{\order}[1]{O( #1 )}

\newcommand{\kron}{\otimes}


\newcommand{\hdr}[2]{
  \pagestyle{fancy}
  \lhead{Bindel, Fall 2019}
  \rhead{Matrix Computation}
  \fancyfoot{}
  \begin{center}
    {\large{\bf HW for #1}} \\ (due: #2)
  \end{center}
  \lstset{language=matlab,columns=flexible}
}


\begin{document}

\hdr{2019-09-11}

\section{Notions of error}

The art of numerics is finding an approximation with a fast algorithm,
a form that is easy to analyze, and an error bound.  Given a task, we
want to engineer an approximation that is good enough, and that
composes well with other approximations.  To make these goals precise,
we need to define types of errors and error propagation, and some
associated notation -- which is the point of this lecture.

\subsection{Absolute and relative error}

Suppose $\hat{x}$ is an approximation to $x$.
The {\em absolute error} is
\[
  e_{\mathrm{abs}} = |\hat{x}-x|.
\]
Absolute error has the same dimensions as $x$,
and can be misleading without some context.  An error
of one meter per second is dramatic if $x$ is my walking
pace; if $x$ is the speed of light, it is a very small error.

The {\em relative error} is a measure with a more natural
sense of scale:
\[
  e_{\mathrm{rel}} = \frac{|\hat{x}-x|}{|x|}.
\]
Relative error is familiar in everyday life: when someone
talks about an error of a few percent, or says that a given
measurement is good to three significant figures, she is describing
a relative error.

We sometimes estimate the relative error in approximating
$x$ by $\hat{x}$ using the relative error in approximating
$\hat{x}$ by $x$:
\[
  \hat{e}_{\mathrm{rel}} = \frac{|\hat{x}-x|}{|\hat{x}|}.
\]
As long as $\hat{e}_{\mathrm{rel}} < 1$, a little algebra gives that
\[
  \frac{\hat{e}_{\mathrm{rel}}}{1+\hat{e}_{\mathrm{rel}}} \leq
  e_{\mathrm{rel}} \leq
  \frac{\hat{e}_{\mathrm{rel}}}{1-\hat{e}_{\mathrm{rel}}}.
\]
If we know $\hat{e}_{\mathrm{rel}}$ is much less than one, then it
is a good estimate for $e_{\mathrm{rel}}$.  If
$\hat{e}_{\mathrm{rel}}$ is not much less than one,
we know that $\hat{x}$ is a poor approximation to $x$.
Either way, $\hat{e}_{\mathrm{rel}}$ is often just as useful
as $e_{\mathrm{rel}}$, and may be easier to estimate.

Relative error makes no sense for $x = 0$, and may be too pessimistic
when the property of $x$ we care about is ``small enough.''  A natural
intermediate between absolute and relative errors is the mixed error
\[
  e_{\mathrm{mixed}} = \frac{|\hat{x}-x|}{|x| + \tau}
\]
where $\tau$ is some natural scale factor associated with $x$.

\subsection{Errors beyond scalars}

Absolute and relative error make sense for vectors as well as scalars.
If $\| \cdot \|$ is a vector
norm and $\hat{x}$ and $x$ are vectors, then the (normwise) absolute
and relative errors are
\begin{align*}
  e_{\mathrm{abs}} &= \|\hat{x}-x\|, &
  e_{\mathrm{rel}} &= \frac{\|\hat{x}-x\|}{\|x\|}.
\end{align*}
We might also consider the componentwise absolute or relative errors
\begin{align*}
  e_{\mathrm{abs},i} &= |\hat{x}_i-x_i| &
  e_{\mathrm{rel},i} &= \frac{|\hat{x}_i-x_i|}{|x_i|}.
\end{align*}
The two concepts are related: the maximum componentwise relative error
can be computed as a normwise error in a norm defined in terms of the
solution vector:
\begin{align*}
  \max_i e_{\mathrm{rel},i} &= \vertiii{\hat{x}-x}
\end{align*}
where $\vertiii{z} = \|\ddiag(x)^{-1} z\|_\infty$.
More generally, absolute error makes sense whenever we can measure
distances between the truth and the approximation; and relative error
makes sense whenever we can additionally measure the size of the
truth.  However, there are often many possible notions of distance
and size; and different ways to measure give different notions of
absolute and relative error.  In practice, this deserves some care.

\subsection{Dimensions and scaling}

The first step in analyzing many application problems is
{\em nondimensionalization}: combining constants in the
problem to obtain a small number of dimensionless constants.
Examples include the aspect ratio of a rectangle,
the Reynolds number in fluid mechanics\footnote{%
Or any of a dozen other named numbers in fluid mechanics.  Fluid
mechanics is a field that appreciates the power of dimensional
analysis}, and so forth.  There are three big reasons to
nondimensionalize:
\begin{itemize}
\item
  Typically, the physics of a problem only really depends on
  dimensionless constants, of which there may be fewer than
  the number of dimensional constants.  This is important
  for parameter studies, for example.
\item
  For multi-dimensional problems in which the unknowns have different
  units, it is hard to judge an approximation error as ``small'' or
  ``large,'' even with a (normwise) relative error estimate.  But one
  can usually tell what is large or small in a non-dimensionalized
  problem.
\item
  Many physical problems have dimensionless parameters much less than
  one or much greater than one, and we can approximate the physics in
  these limits.  Often when dimensionless constants are huge or tiny
  and asymptotic approximations work well, naive numerical methods
  work work poorly.  Hence, nondimensionalization helps us choose how
  to analyze our problems --- and a purely numerical approach may be
  silly.
\end{itemize}

\section{Forward and backward error}

We often approximate a function $f$ by another function $\hat{f}$.
For a particular $x$, the {\em forward} (absolute) error is
\[
  |\hat{f}(x)-f(x)|.
\]
In words, forward error is the function {\em output}.  Sometimes,
though, we can think of a slightly wrong {\em input}:
\[
  \hat{f}(x) = f(\hat{x}).
\]
In this case, $|x-\hat{x}|$ is called the {\em backward} error.
An algorithm that always has small backward error is {\em backward stable}.

A {\em condition number} a tight constant relating relative output
error to relative input error.  For example, for the problem of
evaluating a sufficiently nice function $f(x)$ where $x$ is the input
and $\hat{x} = x+h$ is a perturbed input (relative error $|h|/|x|$),
the condition number $\kappa[f(x)]$ is the smallest constant such that
\[
  \frac{|f(x+h)-f(x)|}{|f(x)|} \leq \kappa[f(x)] \frac{|h|}{|x|} + o(|h|)
\]
If $f$ is differentiable, the condition number is
\[
\kappa[f(x)] =
  \lim_{h \neq 0} \frac{|f(x+h)-f(x)|/|f(x)|}{|(x+h)-x|/|x|} =
  \frac{|f'(x)||x|}{|f(x)|}.
\]
If $f$ is Lipschitz in a neighborhood of $x$ (locally Lipschitz), then
\[
\kappa[f(x)] =
  \frac{M_{f(x)}|x|}{|f(x)|}.
\]
where $M_f$ is the smallest constant such that
$|f(x+h)-f(x)| \leq M_f |h| + o(|h|)$.  When the problem has no linear
bound on the output error relative to the input error, we sat the
problem has an {\em infinite} condition number.  An example is
$x^{1/3}$ at $x = 0$.

A problem with a small condition number is called {\em well-conditioned};
a problem with a large condition number is {\em ill-conditioned}.
A backward stable algorithm applied to a well-conditioned problem has
a small forward error.

\section{Perturbing matrix problems}

To make the previous discussion concrete, suppose I want $y = Ax$, but
because of a small error in $A$ (due to measurement errors or roundoff
effects), I instead compute $\hat{y} = (A+E)x$ where $E$ is ``small.''
The expression for the {\em absolute} error is trivial:
\[
  \|\hat{y}-y\| = \|Ex\|.
\]
But I usually care more about the {\em relative error}.
\[
  \frac{\|\hat{y}-y\|}{\|y\|} = \frac{\|Ex\|}{\|y\|}.
\]
If we assume that $A$ is invertible and that we are using consistent
norms (which we will usually assume), then
\[
  \|Ex\| = \|E A^{-1} y\| \leq \|E\| \|A^{-1}\| \|y\|,
\]
which gives us
\[
  \frac{\|\hat{y}-y\|}{\|y\|} \leq \|A\| \|A^{-1}\|
  \frac{\|E\|}{\|A\|} = \kappa(A) \frac{\|E\|}{\|A\|}.
\]
That is, the relative error in the output is the relative error in the
input multiplied by the condition number
$\kappa(A) = \|A\| \|A^{-1}\|$.
Technically, this is the condition number for the problem of matrix
multiplication (or solving linear systems, as we will see) with
respect to a particular (consistent) norm; different problems have
different condition numbers.  Nonetheless, it is common to call this
``the'' condition number of $A$.

For some problems, we are given more control over the structure of the
error matrix $E$.  For example, we might suppose that $A$ is symmetric,
and ask whether we can get a tighter bound if in addition to assuming a
bound on $\|E\|$, we also assume $E$ is symmetric.  In this particular
case, the answer is ``no'' --- we have the same condition number either
way, at least for the 2-norm or Frobenius norm\footnote{This is left as
an exercise for the student}.  In other cases, assuming a structure to
the perturbation does indeed allow us to achieve tighter bounds.

As an example of a refined bound, we consider moving from condition
numbers based on small norm-wise perturbations to condition numbers
based on small {\em element-wise} perturbations.  Suppose $E$ is
elementwise small relative to $A$, i.e.~$|E| \leq \epsilon |A|$.
Suppose also that we are dealing with a norm such
that $\|X\| \leq \|~|X|~\|$,
as is true of all the norms we have seen so far.  Then
\[
  \frac{\|\hat{y}-y\|}{\|y\|} \leq
  \|EA^{-1}\| \leq \|~|A|~|A^{-1}|~\| \epsilon.
\]
The quantity $\kappa_{\mathrm{rel}}(A) = \|~|A|~|A^{-1}|~\|$ is
the {\em relative condition number};
it is closely related to the {\em Skeel condition number} which we will
see in our discussion of linear systems%
\footnote{The Skeel condition number involves the two factors
in the reverse order.}.
Unlike the standard condition number, the relative condition number is
invariant under column scaling of $A$;
that is $\kappa_{\mathrm{rel}}(AD) = \kappa_{\mathrm{rel}}(A)$
where $D$ is a nonsingular diagonal matrix.

What if, instead of perturbing $A$, we perturb $x$?  That is, if
$\hat{y} = A \hat{x}$ and $y = Ax$, what is the condition number
relating $\|\hat{y}-y\|/\|y\|$ to $\|\hat{x}-x\|/\|x\|$?
We note that
\[
  \|\hat{y}-y\| = \|A(\hat{x}-x)\| \leq \|A\| \|\hat{x}-x\|;
\]
and
\[
  \|x\| = \|A^{-1} y\| \leq \|A^{-1}\| \|y\|
  \quad \implies \quad
  \|y\| \geq \|A^{-1}\|^{-1} \|x\|.
\]
Put together, this implies
\[
  \frac{\|\hat{y}-y\|}{\|y\|} \leq \|A\| \|A^{-1}\|
  \frac{\|\hat{x}-x\|}{\|x\|}.
\]
The same condition number appears again!

\end{document}

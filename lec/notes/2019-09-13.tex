\documentclass[12pt, leqno]{article} %% use to set typesize
\usepackage{fancyhdr}
\usepackage[sort&compress]{natbib}
\usepackage[letterpaper=true,colorlinks=true,linkcolor=black]{hyperref}

\usepackage{amsfonts}
\usepackage{amsmath}
\usepackage{amssymb}
\usepackage{color}
\usepackage{tikz}
\usepackage{pgfplots}
\usepackage{listings}
\usepackage{courier}
%\usepackage[utf8]{inputenc}
%\usepackage[russian]{babel}

\lstset{
  numbers=left,
  basicstyle=\ttfamily\footnotesize,
  numberstyle=\tiny\color{gray},
  stepnumber=1,
  numbersep=10pt,
}

\newcommand{\iu}{\ensuremath{\mathrm{i}}}
\newcommand{\bbR}{\mathbb{R}}
\newcommand{\bbC}{\mathbb{C}}
\newcommand{\calV}{\mathcal{V}}
\newcommand{\calW}{\mathcal{W}}
\newcommand{\macheps}{\epsilon_{\mathrm{mach}}}
\newcommand{\matlab}{\textsc{Matlab}}

\newcommand{\ddiag}{\operatorname{diag}}
\newcommand{\fl}{\operatorname{fl}}
\newcommand{\nnz}{\operatorname{nnz}}
\newcommand{\tr}{\operatorname{tr}}
\renewcommand{\vec}{\operatorname{vec}}

\newcommand{\vertiii}[1]{{\left\vert\kern-0.25ex\left\vert\kern-0.25ex\left\vert #1
    \right\vert\kern-0.25ex\right\vert\kern-0.25ex\right\vert}}
\newcommand{\ip}[2]{\langle #1, #2 \rangle}
\newcommand{\ipx}[2]{\left\langle #1, #2 \right\rangle}
\newcommand{\order}[1]{O( #1 )}

\newcommand{\kron}{\otimes}


\newcommand{\hdr}[2]{
  \pagestyle{fancy}
  \lhead{Bindel, Fall 2019}
  \rhead{Matrix Computation}
  \fancyfoot{}
  \begin{center}
    {\large{\bf HW for #1}} \\ (due: #2)
  \end{center}
  \lstset{language=matlab,columns=flexible}
}


\begin{document}

\hdr{2019-09-13}

\section{Binary floating point}

Binary floating point arithmetic is essentially scientific notation.
Where in decimal scientific notation we write
\[
  \frac{1}{3}
  = 3.333\ldots \times 10^{-1},
\]
in floating point, we write
\[
  \frac{(1)_2}{(11)_2}
  = (1.010101\ldots)_2 \times 2^{-2}.
\]
Because computers are finite, however, we can only keep a finite
number of bits after the binary point.  We can also only keep a finite
number of bits for the exponent field.  These facts turn out to have
interesting implications.

\subsection{Normalized representations}

In general, a {\em normal floating point number} has the form
\[
  (-1)^s \times (1.b_1 b_2 \ldots b_p)_2 \times 2^{E},
\]
where $s \in \{0,1\}$ is the {\em sign bit},
$E$ is the {\em exponent}, and
$(1.b_2 \ldots b_p)_2$ is the {\em significand}.
The normalized representations are called normalized because they
start with a one before the binary point.  Because this is always
the case, we do not need to store that digit explicitly; this gives
us a ``free'' extra digit.

In the 64-bit double precision format, $p = 52$ bits are used to store
the significand, 11 bits are used for the exponent, and one bit is
used for the sign.  The valid exponent range for normal double
precision floating point numbers is $-1023 < E < 1024$; the number
$E$ is encoded as an unsigned binary integer $E_{\mathrm{bits}}$
which is implicitly shifted by 1023 ($E = E_{\mathrm{bits}}-1023$).
This leaves two exponent encodings left over for special purpose,
one associated with $E_{\mathrm{bits}} = 0$ (all bits zero),
and one associated with all bits set; we return to these in a moment.

In the 32-bit single-percision format, $p = 23$ bits are used to store
the significand, 8 bits are used for the exponent, and one bit is used
for the sign.  The valid exponent range for normal is $-127 < E < 128$;
as in the double precision format, the representation is based on an
unsigned integer and an implicit shift, and two bit patterns are left
free for other uses.

We will call the distance between 1.0 and the next largest floating
point number one either an {\em ulp} (unit in the last place) or,
more frequently, machine epsilon (denoted $\macheps$).  This is
$2^{-52} \approx 2 \times 10^{-16}$ for double precision and
$2^{-23} \approx 10^{-7}$ for single precision.  This is the definition
used in most numerical analysis texts, and in MATLAB and Octave, but
it is worth noting that in a few places (e.g.~in the C standard),
call machine epsilon the quantity that is half what we
call machine epsilon.

\subsection{Subnormal representations}

When the exponent field consists of all zero bits, we have a
{\em subnormal} representation.
In general, a {\em subnormal floating point number} has the form
\[
  (-1)^s \times (0.b_1 b_2 \ldots b_p)_2 \times 2^{-E_{\mathrm{bias}}},
\]
where $E_{\mathrm{bias}}$ is 1023 for double precision and 127
for single.  Unlike the normal numbers, the subnormal numbers are
evenly spaced, and so the {\em relative} differences between
successive subnormals can be much larger than the relative differences
between successive normals.

Historically, there have been some floating point systems that lack
subnormal representations; and even today, some vendors encourage
``flush to zero'' mode arithmetic in which all subnormal results are
automatically rounded to zero.  But there are some distinct advantage
to these numbers.  For example, the subnormals allow us to keep the
equivalence between $x-y = 0$ and $x = y$; without subnormals, this
identity can fail to hold in floating point.  Apart from helping us
ensure standard identities, subnormals let us represent numbers close
to zero with reduced accuracy rather than going from full precision to
zero abruptly.  This property is sometimes known as {\em gradual underflow}.

The most important of the subnormal numbers is zero.  In fact, we consider
zero so important that we have two representations: $+0$ and $-0$!
These representations behave the same in most regards, but the sign
does play a subtle role; for example, $1/+0$ gives a representation
for $+\infty$, while $1/-0$ gives a representation for $-\infty$.
The default value of zero is $+0$; this is what is returned, for example,
by expressions such as $1.0-1.0$.

\subsection{Infinities and NaNs}

A floating point representation in which the exponent bits are all
set to one and the signficand bits are all zero represents
an {\em infinity} (positive or negative).

When the exponent bits are all one and the significand bits are not all
zero, we have a {\em NaN} (Not a Number).  A NaN is quiet or signaling
depending on the first bit of the significand; this distinguishes
between the NaNs that simply propagate through arithmetic and those that
cause exceptions when operated upon.  The remaining significand bits
can, in principle, encode information about the details of how and where
a NaN was generated.  In practice, these extra bits are typically
ignored.  Unlike infinities (which can be thought of as a computer
representation of part of the extended reals\footnote{%
The extended reals in this case means $\bbR$ together with $\pm \infty$.
This is sometimes called the {\em two-point compactification} of $\bbR$.
In some areas of analysis (e.g. complex variables),
the {\em one-point compactification} involving a single, unsigned
infinity is also useful.  This was explicitly supported in early
proposals for the IEEE floating point standard, but did not make it in.
The fact that we have signed infinities in floating point is one reason
why it makes sense to have signed zeros --- otherwise, for example,
we would have $1/(1/-\infty)$ yield $+\infty$.
}), NaN ``lives outside'' the extended real numbers.

Infinity and NaN values represent entities that are not part of the
standard real number system.  They should not be interpreted
automatically as ``error values,'' but they should be treated with
respect.  When an infinity or NaN arises in a code
in which nobody has analyzed the code correctness in the presence
of infinity or NaN values, there is likely to be a problem.  But
when they are accounted for in the design and analysis of a floating
point routine, these representations have significant
value.  For example, while an expression like $0/0$ cannot be
interpreted without context (and therefore yields a NaN in floating
point), given context --- eg., a computation involving a removable
singularity --- we may be able to interpret a NaN, and potentially
replace it with some ordinary floating point value.

\section{Basic floating point arithmetic}

For a general real number $x$, we will write
\[
  \fl(x) = \mbox{ correctly rounded floating point representation of } x.
\]
By default, ``correctly rounded'' means that we find the closest
floating point number to $x$, breaking any ties by rounding to the
number with a zero in the last bit\footnote{%
There are other rounding modes beside the default, but we will not
discuss them in this class}.
If $x$ exceeds the largest normal floating point number,
then $\fl(x) = \infty$; similarly, if $x$ is a negative number
with magnitude greater than the most negative normalized floating
point value, then $\fl(x) = -\infty$.

% DSB: Check eps vs eps/2
For basic operations (addition, subtraction, multiplication,
division, and square root), the floating point standard specifies that
the computer should produce the {\em true result, correctly rounded}.
So the MATLAB statement
\lstset{language=matlab,columns=flexible}
\begin{lstlisting}
    % Compute the sum of x and y (assuming they are exact)
    z = x + y;
\end{lstlisting}
actually computes the quantity $\hat{z} = \fl(x+y)$.  If $\hat{z}$ is
a normal double-precision floating point number, it will agree with
the true $z$ to 52 bits after the binary point.  That is, the relative
error will be smaller in magnitude than the {\em machine epsilon}
$\macheps = 2^{-53} \approx 1.1 \times 10^{-16}$:
\[
  \hat{z} = z(1+\delta), \quad |\delta| < \macheps.
\]
More generally, basic operations that produce normalized numbers are
correct to within a relative error of $\macheps$.

The floating
point standard also {\em recommends} that common transcendental functions,
such as exponential and trig functions, should be correctly rounded\footnote{%
For algebraic functions, it is possible to determine in advance how many
additional bits of precision are needed to correctly round the result
for a function of one input.  In contrast, transcendental functions
can produce outputs that fall arbitrarily close to the halfway point
between two floating point numbers.
},
though compliant implementations that do not follow with this
recommendation may produce results with a relative error
just slightly larger than $\macheps$.
Correct rounding of transcendentals is useful in large part because it
implies other properties: for example, if a computer function to evaluate
a monotone function returns a correctly rounded result, then the computed
function is also monotone.

Operations in which NaN appears as an input conventionally (but not always)
produce a NaN output.  Comparisons in which NaN appears conventionally produce false.
But sometimes there is some subtlety in accomplishing these
semantics.  For example, the following code for finding the maximum
element of a vector returns a NaN if one appears
in the first element, but otherwise results in the largest non-NaN
element of the array:
\lstinputlisting{code/mymax1.m}
In contrast, the following code always propagates a NaN to the output
if one appears in the input
\lstinputlisting{code/mymax2.m}
You are encouraged to play with different vectors involving some NaN
or all NaN values to see what the semantics for the built-in
vector max are in MATLAB, Octave, or your language of choice.
You may be surprised by the results!

Apart from NaN, floating point numbers do correspond to real numbers,
and comparisons between floating point numbers have the usual semantics
associated with comparisons between floating point numbers.  The only
point that deserves some further comment is that plus zero and minus
zero are considered equal as floating point numbers, despite the fact
that they are not bitwise identical (and do not produce identical
results in all input expressions)\footnote{%
This property of signed zeros is just a little bit horrible.
But to misquote Winston Churchill, it is the worst
definition of equality except all the others that have been tried.
}.

\section{Exceptions}

We say there is an {\em exception} when the floating point result is
not an ordinary value that represents the exact result.  The most
common exception is {\em inexact} (i.e. some rounding was needed).
Other exceptions occur when we fail to produce a normalized floating
point number.  These exceptions are:
\begin{description}
\item[Underflow:]
  An expression is too small to be represented as a normalized floating
  point value.  The default behavior is to return a subnormal.
\item[Overflow:]
  An expression is too large to be represented as a floating point
  number.  The default behavior is to return {\tt inf}.
\item[Invalid:]
  An expression evaluates to Not-a-Number (such as $0/0$)
\item[Divide by zero:]
  An expression evaluates ``exactly'' to an infinite value
  (such as $1/0$ or $\log(0)$).
\end{description}
When exceptions other than inexact occur, the usual ``$1 + \delta$''
model used for most rounding error analysis is not valid.

An important feature of the floating point standard is that an
exception should {\em not} stop the computation by default.  This
is part of why we have representations for infinities and NaNs:
the floating point system is {\em closed} in the sense that every
floating point operation will return some result in the floating
point system.  Instead, by default, an exception is {\em flagged}
as having occurred\footnote{%
There is literally a register inside the computer with a set of
flags to denote whether an exception has occurred in a given
chunk of code.  This register is highly problematic, as it
represents a single, centralized piece of global state.  The
treatment of the exception flags --- and of exceptions generally ---
played a significant role in the debates leading up to the last revision
of the IEEE 754 floating point standard,
and I would be surprised if they are not playing a role again in the
current revision of the standard.
}.  An actual exception (in the sense of hardware
or programming language exceptions) occurs only if requested.

\section{Modeling floating point}

The fact that normal floating point results have a relative error
bounded by $\macheps$ gives us a useful {\em model} for reasoning about
floating point error.  We will refer to this as the ``$1 + \delta$''
model.  For example, suppose $x$ is an exactly-represented input to
the MATLAB statement
\begin{lstlisting}
    z = 1-x*x;
\end{lstlisting}
We can reason about the error in the computed $\hat{z}$ as follows:
\begin{align*}
  t_1 &= \fl(x^2) = x^2 (1+\delta_1) \\
  t_2 &= 1-t_1 = (1-x^2)\left( 1 - \frac{\delta_1 x^2}{1-x^2} \right) \\
  \hat{z}
  &= \fl(1-t_1)
    = z \left( 1 - \frac{\delta_1 x^2}{1-x^2} \right)(1+\delta_2) \\
  & \approx z \left( 1 - \frac{\delta_1 x^2}{1-x^2} +\delta_2 \right),
\end{align*}
where $|\delta_1|, |\delta_2| \leq \macheps$.  As before, we throw
away the (tiny) term involving $\delta_1 \delta_2$.
Note that if $z$ is close to zero (i.e.~if there is {\em cancellation} in the
subtraction), then the model shows the result may have a
large relative error.

\subsection{First-order error analysis}

Analysis in the $1+\delta$ model quickly gets to be a sprawling mess
of Greek letters unless one is careful.  A standard trick to get
around this is to use {\em first-order} error analysis in which we
linearize all expressions involving roundoff errors.  In particular,
we frequently use the approximations
\begin{align*}
  (1+\delta_1)(1+\delta_2) & \approx 1+\delta_1 + \delta_2 \\
  1/(1+\delta) & \approx 1-\delta.
\end{align*}
In general, we will resort to first-order analysis without comment.
Those students who think this is a sneaky trick to get around our
lack of facility with algebra\footnote{%
Which it is.
}
may take comfort in the fact that if $|\delta_i| < \macheps$, then
in double precision
\[
  \left| \prod_{i=1}^n (1+\delta_i) \prod_{i=n+1}^N (1+\delta_i)^{-1} \right| < (1+1.03 N \macheps)
\]
for $N < 10^{14}$ (and a little further).

\subsection{Shortcomings of the model}

The $1+\delta$ model has two shortcomings.  First, it is only valid
for expressions that involve normalized numbers --- most notably,
gradual underflow breaks the model.  Second, the model is sometimes
pessimistic.  Certain operations, such as taking a difference between
two numbers within a factor of $2$ of each other, multiplying or
dividing by a factor of two\footnote{Assuming that the result
does not overflow or produce a subnormal.}, or multiplying two
single-precision numbers into a double-precision result,
are {\em exact} in floating point.  There are useful operations
such as simulating extended precision using ordinary floating point
that rely on these more detailed properties of the floating point system,
and cannot be analyzed using just the $1+\delta$ model.


\end{document}

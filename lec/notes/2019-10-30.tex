\documentclass[12pt, leqno]{article} %% use to set typesize
\usepackage{fancyhdr}
\usepackage[sort&compress]{natbib}
\usepackage[letterpaper=true,colorlinks=true,linkcolor=black]{hyperref}

\usepackage{amsfonts}
\usepackage{amsmath}
\usepackage{amssymb}
\usepackage{color}
\usepackage{tikz}
\usepackage{pgfplots}
\usepackage{listings}
\usepackage{courier}
%\usepackage[utf8]{inputenc}
%\usepackage[russian]{babel}

\lstset{
  numbers=left,
  basicstyle=\ttfamily\footnotesize,
  numberstyle=\tiny\color{gray},
  stepnumber=1,
  numbersep=10pt,
}

\newcommand{\iu}{\ensuremath{\mathrm{i}}}
\newcommand{\bbR}{\mathbb{R}}
\newcommand{\bbC}{\mathbb{C}}
\newcommand{\calV}{\mathcal{V}}
\newcommand{\calW}{\mathcal{W}}
\newcommand{\macheps}{\epsilon_{\mathrm{mach}}}
\newcommand{\matlab}{\textsc{Matlab}}

\newcommand{\ddiag}{\operatorname{diag}}
\newcommand{\fl}{\operatorname{fl}}
\newcommand{\nnz}{\operatorname{nnz}}
\newcommand{\tr}{\operatorname{tr}}
\renewcommand{\vec}{\operatorname{vec}}

\newcommand{\vertiii}[1]{{\left\vert\kern-0.25ex\left\vert\kern-0.25ex\left\vert #1
    \right\vert\kern-0.25ex\right\vert\kern-0.25ex\right\vert}}
\newcommand{\ip}[2]{\langle #1, #2 \rangle}
\newcommand{\ipx}[2]{\left\langle #1, #2 \right\rangle}
\newcommand{\order}[1]{O( #1 )}

\newcommand{\kron}{\otimes}


\newcommand{\hdr}[2]{
  \pagestyle{fancy}
  \lhead{Bindel, Fall 2019}
  \rhead{Matrix Computation}
  \fancyfoot{}
  \begin{center}
    {\large{\bf HW for #1}} \\ (due: #2)
  \end{center}
  \lstset{language=matlab,columns=flexible}
}


\newcommand{\uQ}{\underline{Q}}
\newcommand{\uR}{\underline{R}}

\begin{document}

\hdr{2019-10-30}

\section{Hessenberg matrices and QR steps in $O(n^2)$}

A matrix $H$ is said to be {\em upper Hessenberg} if it has
nonzeros only in the upper triangle and the first subdiagonal.
For example, the nonzero structure of a 5-by-5 Hessenberg matrix
is
\[
  \begin{bmatrix}
    \times & \times & \times & \times & \times \\
    \times & \times & \times & \times & \times \\
           & \times & \times & \times & \times \\
           &        & \times & \times & \times \\
           &        &        & \times & \times
  \end{bmatrix}.
\]
For any square matrix $A$, we can find a unitarily similar Hessenberg
matrix $H = Q^* A Q$ by the following algorithm:
\lstinputlisting{code/eigen/hessred.m}

A Hessenberg matrix $H$ is very nearly upper triangular, and is an
interesting object in its own right for many applications.  For
example, in control theory, one sometimes would like to evaluate a
{\em transfer function}
\[
  h(s) = c^T (sI-A)^{-1} b + d
\]
for many different values of $s$.  Done naively, it looks like each
each evaluation would require $O(n^3)$ time in order to get a
factorization of $sI-A$; but if $H = Q^* A Q$ is upper Hessenberg, we
can write
\[
  h(s) = (Qc)^* (sI-H)^{-1} (Qb) + d,
\]
and the Hessenberg structure of $sI-H$ allows us to do Gaussian
elimination on it in $O(n^2)$ time.

Just as it makes it cheap to do Gaussian elimination, the special
structure of the Hessenberg matrix also makes the Householder QR
routine very economical.  The Householder reflection computed in order
to introduce a zero in the $(j+1,j)$ entry needs only to operate on
rows $j$ and $j+1$.  Therefore, we have
\[
  Q^* H = W_{n-1} W_{n-2} \ldots W_1 H = R,
\]
where $W_{j}$ is a Householder reflection that operates only on rows
$j$ and $j+1$.  Computing $R$ costs $O(n^2)$ time, since each $W_j$
only affects two rows ($O(n)$ data).  Now, note that
\[
  R Q = R (W_1 W_2 \ldots W_{n-1});
\]
that is, $RQ$ is computed by an operation that first mixes the first
two columns, then the second two columns, and so on.  The only subdiagonal
entries that can be introduced in this process lie on the first subdiagonal,
and so $RQ$ is again a Hessenberg matrix.  Therefore, one step of QR iteration
on a Hessenberg matrix results in another Hessenberg matrix, and a Hessenberg
QR step can be performed in $O(n^2)$ time.

Putting these ideas in concrete form, we have the following code
\lstinputlisting{code/eigen/hessqr_basic.m}

\section{Inverse iteration and the QR method}

When we discussed the power method, we found that we could improve
convergence by a spectral transformation that mapped the eigenvalue we
wanted to something with large magnitude (preferably much larger than
the other eigenvalues).  This was the {\em shift-invert} strategy.
We already know there is a connection leading from the power method
to orthogonal iteration to the QR method, which we can summarize with
a small number of formulas.  Let us see if we can follow the same
path to uncover a connection from inverse iteration (the power method
with $A^{-1}$, a special case of shift-invert in which the shift is zero) to QR.
If we call the orthogonal factors
in orthogonal iteration  $\uQ^{(k)}$ ($\uQ^{(0)} = I$) and the iterates
in QR iteration $A^{(k)}$, we have
\begin{align}
  A^{k}   &= \uQ^{(k)} \uR^{(k)} \label{orth-iter-rel} \\
  A^{(k)} &= (\uQ^{(k)})^* A (\uQ^{(k)}).
\end{align}
In particular, note that because $R^{(k)}$ are upper triangular,
\[
  A^{k} e_1 = (\uQ^{(k)} e_1) r^{(k)}_{11};
\]
that is, the first column of $\uQ^{(k)}$ corresponds to the $k$th
step of power iteration starting at $e_1$.  What happens when we
consider negative powers of $A$?  Inverting (\ref{orth-iter-rel}),
we find
\[
  A^{-k} = (\uR^{(k)})^{-1} (\uQ^{(k)})^*
\]
The matrix $\tilde{R}^{(k)} = (\uR^{(k)})^{-1}$ is again upper triangular;
and if we look carefully, we can see in this fact another power iteration:
\[
  e_n^* A^{-k} = e_n^* \tilde{R}^{(k)} (\uQ^{(k)})^*
              = \tilde{r}^{(k)}_{nn} (\uQ^{(k)} e_n)^*.
\]
That is, the last column of $\uQ^{(k)}$ corresponds to a power iteration
converging to a {\em row} eigenvector of $A^{-1}$.

\section{Shifting gears}

The connection from inverse iteration to orthogonal iteration (and
thus to QR iteration) gives us a way to incorporate the shift-invert
strategy into QR iteration: simply run QR on the matrix $A-\sigma I$,
and the $(n,n)$ entry of the iterates (which corresponds to a Rayleigh
quotient with an increasingly-good approximate row eigenvector) should
start to converge to $\lambda - \sigma$, where $\lambda$ is the
eigenvalue nearest $\sigma$.  Put differently, we can run the
iteration:
\begin{align*}
  Q^{(k)} R^{(k)} &= A^{(k-1)} - \sigma I \\
  A^{(k)} &= R^{(k)} Q^{(k)} + \sigma I.
\end{align*}
If we choose a good shift, then the lower right corner entry of
$A^{(k)}$ should converge to the eigenvalue closest to $\sigma$ in
fairly short order, and the rest of the elements in the last row
should converge to zero.

The shift-invert power iteration converges fastest when we choose a
shift that is close to the eigenvalue that we want.  We can do even
better if we choose a shift {\em adaptively}, which was the basis for
running Rayleigh quotient iteration.  The same idea is the basis
for the {\em shifted QR iteration}:
\begin{align}
  Q^{(k)} R^{(k)} &= A^{(k-1)} - \sigma_k I \label{sqr-1} \\
  A^{(k)} &= R^{(k)} Q^{(k)} + \sigma_k I. \label{sqr-2}
\end{align}
This iteration is equivalent to computing
\begin{align*}
  \uQ^{(k)} \uR^{(k)} &= \prod_{j=1}^n (A-\sigma_j I) \\
  A^{(k)} &= (\uQ^{(k)})^* A (\uQ^{(k)}) \\
  \uQ^{(k)} &= Q^{(k)} Q^{(k-1)} \ldots Q^{(1)}.
\end{align*}

What should we use for the shift parameters $\sigma_k$?  A natural
choice is to use $\sigma_k = e_n^* A^{(k-1)} e_n$, which is the same
as $\sigma_k = (\uQ^{(k)} e_n)^* A (\uQ^{(k)} e_n)$, the Rayleigh quotient
based on the last column of $\uQ^{(k)}$.  This simple shifted QR iteration
is equivalent to running Rayleigh iteration starting from an initial vector
of $e_n$, which we noted before is locally quadratically convergent.

\end{document}

\documentclass[12pt, leqno]{article} %% use to set typesize
\usepackage{fancyhdr}
\usepackage[sort&compress]{natbib}
\usepackage[letterpaper=true,colorlinks=true,linkcolor=black]{hyperref}

\usepackage{amsfonts}
\usepackage{amsmath}
\usepackage{amssymb}
\usepackage{color}
\usepackage{tikz}
\usepackage{pgfplots}
\usepackage{listings}
\usepackage{courier}
%\usepackage[utf8]{inputenc}
%\usepackage[russian]{babel}

\lstset{
  numbers=left,
  basicstyle=\ttfamily\footnotesize,
  numberstyle=\tiny\color{gray},
  stepnumber=1,
  numbersep=10pt,
}

\newcommand{\iu}{\ensuremath{\mathrm{i}}}
\newcommand{\bbR}{\mathbb{R}}
\newcommand{\bbC}{\mathbb{C}}
\newcommand{\calV}{\mathcal{V}}
\newcommand{\calW}{\mathcal{W}}
\newcommand{\macheps}{\epsilon_{\mathrm{mach}}}
\newcommand{\matlab}{\textsc{Matlab}}

\newcommand{\ddiag}{\operatorname{diag}}
\newcommand{\fl}{\operatorname{fl}}
\newcommand{\nnz}{\operatorname{nnz}}
\newcommand{\tr}{\operatorname{tr}}
\renewcommand{\vec}{\operatorname{vec}}

\newcommand{\vertiii}[1]{{\left\vert\kern-0.25ex\left\vert\kern-0.25ex\left\vert #1
    \right\vert\kern-0.25ex\right\vert\kern-0.25ex\right\vert}}
\newcommand{\ip}[2]{\langle #1, #2 \rangle}
\newcommand{\ipx}[2]{\left\langle #1, #2 \right\rangle}
\newcommand{\order}[1]{O( #1 )}

\newcommand{\kron}{\otimes}


\newcommand{\hdr}[2]{
  \pagestyle{fancy}
  \lhead{Bindel, Fall 2019}
  \rhead{Matrix Computation}
  \fancyfoot{}
  \begin{center}
    {\large{\bf HW for #1}} \\ (due: #2)
  \end{center}
  \lstset{language=matlab,columns=flexible}
}


\newcommand{\uQ}{\underline{Q}}
\newcommand{\uR}{\underline{R}}

\begin{document}

\hdr{2019-10-28}

\section{Orthogonal iteration revisited}

Last time, we described a generalization of the power methods to
compute invariant subspaces.  That is, starting from some initial
subspace $\calV^{(0)}$, we defined the sequence
\[
  \calV^{(k+1)} = A \calV^{(k)} = A^{k+1} \calV^{(0)}.
\]
Under some assumptions, the spaces $\calV^{(k)}$ asymptotically converge
to an invariant subspace of $A$.
In order to actually implement this iteration, though, we need
a concrete representation of each of the subspaces in terms of a basis.
In the case of the power method, we normalized our vector at each step
to have unit length.  The natural generalization in the case of
subspace iteration is to normalize our bases to be orthonormal at each
step.  If the columns of $V^{(k)}$ form an orthonormal basis for
$\calV^{(k)}$, then the columns of $AV^{(k)}$ form an orthonormal
basis for $\calV^{(k+1)}$; and we can compute an orthonormal basis
$V^{(k+1)}$ for $\calV^{(k+1)}$ by an economy QR decomposition:
\[
  V^{(k+1)} R^{(k+1)} = A V^{(k)}.
\]
This {\em orthogonal iteration} gives us a sequence of orthonormal
bases $V^{(k+1)}$ for the spaces $\calV^{(k)}$.

We also mentioned in the last lecture that orthogonal iteration has
the marvelous property that it runs subspace iteration
{\em for a sequence of nested subspaces}.  Using MATLAB notation,
we have that for any $l$,
\[
  V^{(k+1)}(:,1:l) R^{(k+1)}(1:l,1:l) = A V^{(k)}(:,1:l).
\]
So by running orthogonal iteration on an $m$-dimensional subspace,
we magically {\em also} run orthogonal iteration on an $l$-dimensional
subspaces for each $l \leq m$.  Recall that the Schur factorization
\[
  AU = UT
\]
involves an orthonormal basis $U$ such that for any $l$,
$U(:,1:l)$ spans an $l$-dimensional invariant subspace
(this is from the triangularity of $T$).  So we might hope that
if we ran orthogonal iteration on {\em all} of $A$, we would
eventually converge to the $U$ matrix in the Schur factorization.
That is, starting from $\uQ^{(0)} = I$, we iterate
\begin{equation} \label{orth-iter-def}
  \uQ^{(k+1)} R^{(k+1)} = A \uQ^{(k)}
\end{equation}
in the hopes that the columns of $\uQ^{(k)}$, since they span
nested bases undergoing subspace iteration, will converge
to the unitary factor $U$.

Now, consider the first two steps of this iteration:
\begin{align*}
  \uQ^{(1)} R^{(1)} &= A \\
  \uQ^{(2)} R^{(2)} &= A \uQ^{(1)}
\end{align*}
If we multiply the second equation on the right by $R^{(1)}$,
we have
\begin{equation} \label{orth-iter-2}
  \uQ^{(2)} R^{(2)} R^{(1)} = A \uQ^{(1)} R^{(1)} = A^2.
\end{equation}
Similarly, if we multiply $\uQ^{(3)} R^{(3)} = A \uQ^{(3)}$ by $R^{(2)} R^{(1)}$,
we have
\[
  \uQ^{(3)} R^{(3)} R^{(2)} R^{(1)} = A \uQ^{(2)} R^{(2)} R^{(1)} = A^3,
\]
where the last equality comes from (\ref{orth-iter-2}).
We can keep going in this fashion, and if we define the
upper triangular matrix
$\uR^{(k)} = R^{(k)} R^{(k-1)} \ldots R^{(1)}$, we have
\[
  \uQ^{(k)} \uR^{(k)} = A^k.
\]
That is, $\uQ^{(k)}$ is precisely the unitary factor in a QR
decomposition of $A^k$.  This fact may be unsurprising if
we consider that we derived this orthogonal iteration from
the power method.

\section{Orthogonal iteration to QR}

The focus of orthogonal iteration is the orthogonal (or unitary)
factor in the Schur form.  The upper triangular Schur factor
fades into the background, which is a pity; this is, after all, where we
learn the eigenvalues.  But through a bit of algebraic trickery, it turns
out that we can re-invent orthogonal iteration as an iteration for
the upper triangular factor.  The steps for this transformation are as
follows:
\begin{enumerate}
\item
  The orthogonal iteration $\uQ^{(k+1)} R^{(k)} = A \uQ^{(k)}$ is a
  generalization of the power method.  In fact, the first column of
  this iteration is {\em exactly} the power iteration.  In general,
  the first $p$ columns of $\uQ^{(k)}$ are converging to an orthonormal
  basis for a $p$-dimensional invariant subspace associated with the
  $p$ eigenvalues of $A$ with largest modulus (assuming that there
  aren't several eigenvalues with the same modulus to make this
  ambiguous).
\item
  If all the eigenvalues have different modulus, orthogonal iteration
  ultimately converges to the orthogonal factor in a Schur form
  \[
    AU = UT
  \]
  What about the $T$ factor?  Note that $T = U^* A U$, so a natural
  approximation to $T$ at step $k$ would be
  \[
    A^{(k)} = (\uQ^{(k)})^* A \uQ^{(k)},
  \]
  and from the definition of the subspace iteration, we have
  \[
    A^{(k)} = (\uQ^{(k)})^* \uQ^{(k+1)} R^{(k)} = Q^{(k)} R^{(k)},
  \]
  where $Q^{(k)} \equiv (\uQ^{(k)})^* \uQ^{(k+1)}$ is unitary.
\item
  Note that
  \[
    A^{(k+1)}
    = (\uQ^{(k+1)})^* A^{(k)} \uQ^{(k+1)}
    = (Q^{(k)})^* A^{(k)} Q^{(k)}
    = R^{(k)} Q^{(k)}.
  \]
  Thus, we can go from $A^{(k)}$ to $A^{(k+1)}$ directly without
  the orthogonal factors from subspace iteration, simply by computing
  \begin{align*}
    A^{(k)} &= Q^{(k)} R^{(k)} \\
    A^{(k+1)} &= R^{(k)} Q^{(k)}.
  \end{align*}
  This is the {\em QR iteration}.
\end{enumerate}

Under some restrictions on $A$, the matrices $A^{(k)}$ will ultimately
converge to the triangular Schur factor of $A$.  But we have two problems:
\begin{enumerate}
\item
  Each step of the QR iteration requires a QR factorization, which is
  an $O(n^3)$ operation.  This is rather expensive, and even in the happy
  case where we might be able to get each eigenvalue with a constant
  number of steps, $O(n)$ total steps at a cost of $O(n^3)$ each gives
  us an $O(n^4)$ algorithm.  Given that everything else we have done
  so far costs only $O(n^3)$, an $O(n^4)$ cost for eigenvalue computation
  seems excessive.
\item
  Like the power iteration upon which it is based, the basic iteration
  converges linearly, and the rate of convergence is related to the
  ratios of the moduli of eigenvalues.  Convergence is slow when there
  are eigenvalues of nearly the same modulus, and nonexistent when
  there are eigenvalues with the same modulus.
\end{enumerate}
We describe next how to overcome these two difficulties.

\end{document}

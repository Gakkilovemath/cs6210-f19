\documentclass[12pt, leqno]{article} %% use to set typesize
\usepackage{fancyhdr}
\usepackage[sort&compress]{natbib}
\usepackage[letterpaper=true,colorlinks=true,linkcolor=black]{hyperref}

\usepackage{amsfonts}
\usepackage{amsmath}
\usepackage{amssymb}
\usepackage{color}
\usepackage{tikz}
\usepackage{pgfplots}
\usepackage{listings}
\usepackage{courier}
%\usepackage[utf8]{inputenc}
%\usepackage[russian]{babel}

\lstset{
  numbers=left,
  basicstyle=\ttfamily\footnotesize,
  numberstyle=\tiny\color{gray},
  stepnumber=1,
  numbersep=10pt,
}

\newcommand{\iu}{\ensuremath{\mathrm{i}}}
\newcommand{\bbR}{\mathbb{R}}
\newcommand{\bbC}{\mathbb{C}}
\newcommand{\calV}{\mathcal{V}}
\newcommand{\calW}{\mathcal{W}}
\newcommand{\macheps}{\epsilon_{\mathrm{mach}}}
\newcommand{\matlab}{\textsc{Matlab}}

\newcommand{\ddiag}{\operatorname{diag}}
\newcommand{\fl}{\operatorname{fl}}
\newcommand{\nnz}{\operatorname{nnz}}
\newcommand{\tr}{\operatorname{tr}}
\renewcommand{\vec}{\operatorname{vec}}

\newcommand{\vertiii}[1]{{\left\vert\kern-0.25ex\left\vert\kern-0.25ex\left\vert #1
    \right\vert\kern-0.25ex\right\vert\kern-0.25ex\right\vert}}
\newcommand{\ip}[2]{\langle #1, #2 \rangle}
\newcommand{\ipx}[2]{\left\langle #1, #2 \right\rangle}
\newcommand{\order}[1]{O( #1 )}

\newcommand{\kron}{\otimes}


\newcommand{\hdr}[2]{
  \pagestyle{fancy}
  \lhead{Bindel, Fall 2019}
  \rhead{Matrix Computation}
  \fancyfoot{}
  \begin{center}
    {\large{\bf HW for #1}} \\ (due: #2)
  \end{center}
  \lstset{language=matlab,columns=flexible}
}


\begin{document}

\hdr{2019-11-20}

\section{Quasi-optimality}

We quantify the stability of a subspace approximation method via
a {\em quasi-optimality bound}:
\[
  \|x^*-\hat{x}\| \leq C \min_{v \in \calV} \|x^*-v\|.
\]
That is, the approximation $\hat{x}$ is quasi-optimal if it has
error within some factor $C$ of the best error possible within the
space.

To derive quasi-optimality results, it is useful to think of all
of our methods as defining a {\em solution projector}
that maps $x^*$ to the approximate solution to $A\hat{x} = Ax^* = b$.
From the (Petrov-)Galerkin perspective, if $W \in \bbR^{n \times k}$
and $V \in \bbR^{n \times k}$ are bases for the trial space $\calW$
and $\calV$, respectively, then we have
\begin{align*}
  W^T A V \hat{y} &= W^T b, \quad \hat{x} = V \hat{y} \\
  \hat{x} &= V (W^T A V)^{-1} W^T b \\
          &= V (W^T A V)^{-1} W^T A x^*. \\
          &= \Pi x^*.
\end{align*}
The {\em error projector} $I-\Pi$ maps $x^*$ to the error $\hat{x}-x^*$
in approximately solving $A\hat{x} \approx Ax^* = b$.  There is no
error iff $x^*$ is actually in $\calV$; that is, $\calV$ is the
null space of $I-\Pi$.  Hence, if $\tilde{x}$ is any vector in $\calV$,
then
\[
  \hat{e} = (I-\Pi) x = (I-\Pi) (x-\tilde{x}) = (I-\Pi) \tilde{e}.
\]
Therefore we have
\[
  \|x-\hat{x}\| \leq \|I-\Pi\| \min_{\tilde{x} \in \calV} \|x-\tilde{x}\|,
\]
and a bound on $\|I-\Pi\|$ gives a quasi-optimality result.

For any operator norm, we have
\[
  |I-\Pi\| \leq 1+\|\Pi\| \leq 1 + \|V\| \|(W^T A V)^{-1}\| \|W^T A\|;
\]
and in any Euclidean norm, if $V$ and $W$ are chosen to have orthonormal
columns, then
\[
  \|I-\Pi\| \leq 1 + \|(W^T A V)^{-1}\| \|A\|.
\]
If $A$ is symmetric and positive definite and $V = W$, then the
interlace theorem gives $\|(V^T A V)^{-1}\| \leq \|A^{-1}\|$,
and the quasi-optimality constant is bounded by $1 + \kappa(A)$.
In more general settings, though, we may have no guarantee that
the projected matrix $W^T A V$ is far from singular, even if $A$
itself is nonsingular.  To guarantee boundedness of $(W^T A V)^{-1}$
{\em a priori} requires a compatibility condition relating
$\calW$, $\calV$, and $A$; such a condition is sometimes called
the {\em LBB} condition
(for Ladyzhenskaya-Babu\v{s}ka-Brezzi) or
the {\em inf-sup} condition, so named because (as we have discussed
previously)
\[
  \sigma_{\min}(W^T A V) =
  \inf_{w \in \calW} \sup_{v \in \calV} \frac{w^T A v}{\|w\| \|v\|}.
\]
The LBB condition plays an important role when Galerkin methods are
used to solve large-scale PDE problems, since there it is easy to
choose the spaces $\calV$ and $\calW$ in a way that leads to very
bad conditioning.  But for iterative solvers of the type we discuss
in this course (Krylov subspace solvers), such pathologies are a more
rare occurrence.  In this setting, we may prefer to
monitor $\|(W^T A V)^{-1}\|$ directly as we go along, and to simply
increase the dimension of the space if we ever run into trouble.

\section{Model reduction}

Our focus in this section is methods for solving a single linear
system at a time.  Often, we want to solve many closely-related
linear systems with different matrices.  As a simple example,
we might want to evaluate
\[
  (A-\sigma I) x(\sigma) = b
\]
for several different values of $\sigma$ within some range; more
generally, we might want to solve linear systems $A(s) x(s) = b(s)$
where $A$ and $b$ depend smoothly on some low-dimensional parameter
vector $s$ that varies over a bounded set.  In such settings, one often
finds (and can sometimes prove via interpolation theory) that $x(s)$
lies close to a space $\calV$ that can be computed. For example, we
might find that an adequate space $\calV$ spanned by sample solutions
$x(s_1), x(s_2), \ldots$; we could then choose a corresponding trial
space $\calW$ as the basis for a Galerkin scheme.  Hence, we may
estimate $x(\sigma)$ very quickly (online) after a more expensive
computation to construct a basis for an appropriate approximation space
(offline).

There are a wide-variety of techniques that employ this general idea.
These include model reduction methods from control theory
(moment-matching methods that use Krylov subspaces, truncated balanced
realization methods that involve solving Sylvester equations, etc);
global-basis methods for the solution of PDEs (e.g.~the so-called {\em
empirical interpolation method}); and many other methods for both
linear and nonlinear problems.  While it is not a focus for this course,
the approach is so simple and broadly applicable that I would feel bad
if you left not knowing about it.

\section{Krylov subspaces}

The {\em Krylov subspace} of dimension $k$ generated by
$A \in \bbR^{n \times n}$ and $b \in \bbR^n$ is
\[
  \mathcal{K}_k(A,b)
    = \operatorname{span}\{ b, Ab, \ldots, A^{k-1} b \}
    = \{ p(A) b : p \in \mathcal{P}_{k-1} \}.
\]
Krylov subspaces are a natural choice for subspace-based methods for
approximate linear solves, for two reasons:
\begin{itemize}
\item If all you are allowed to do with $A$ is compute matrix-vector
  products, and the only vector at hand is $b$, what else would you do?
\item The Krylov subspaces have excellent approximation properties.
\end{itemize}

Krylov subspaces have several properties that are worthy of comment.
Because the vectors $A^{j} b$ are proportional to the vectors obtained
in power iteration, one might reasonably (and correctly)
assume that the space quickly contains good approximations to the
eigenvectors associated with the largest magnitude eigenvalues.
Krylov subspaces are also {\em shift-invariant}, i.e. for any $\sigma$
\[
  \mathcal{K}_k(A-\sigma I, b) = \mathcal{K}_k(A,b).
\]
By choosing different shifts, we can see that the Krylov subspaces
tend to quickly contain not only good approximations to the eigenvector
associated with the largest magnitude eigenvalue, but to all
``extremal'' eigenvalues.

Most arguments about the approximation properties of Krylov subspaces
derive from the characterization of the space as all vectors $p(A) b$
where $p \in \mathcal{P}_{k-1}$ and from the spectral mapping theorem,
which says that if $A = V \Lambda V^{-1}$ then
$p(A) = V p(\Lambda) V^{-1}$.  Hence, the distance between
an arbitrary vector (say $d$) and the Krylov subspace is
\[
  \min_{p \in \mathcal{P}_{k-1}}
  \left\| V \left[ p(\Lambda) V^{-1} b - V^{-1} d \right] \right\|.
\]
As a specific example, suppose that we want to choose $\hat{x}$
in a Krylov subspace in order to minimize the residual $A \hat{x} - b$.
Writing $\hat{x} = p(A) b$, we have that we want to minimize
\[
  \|[A p(A)-I] b\| = \|q(A) b\|
\]
where $q(z)$ is a polynomial of degree at most $k$ such that $q(1) = 1$.
The best possible residual in this case is bounded by
\[
  \|q(A) b\| \leq \kappa(V) \|q(\Lambda)\| \|b\|,
\]
and so the relative residual can be bounded in terms of the condition
number of $V$ and the minimum value that can bound $q$ on the spectrum
of $A$ subject to the constraint that $q(0) = 1$.

\end{document}

\documentclass[12pt, leqno]{article} %% use to set typesize
\usepackage{fancyhdr}
\usepackage[sort&compress]{natbib}
\usepackage[letterpaper=true,colorlinks=true,linkcolor=black]{hyperref}

\usepackage{amsfonts}
\usepackage{amsmath}
\usepackage{amssymb}
\usepackage{color}
\usepackage{tikz}
\usepackage{pgfplots}
\usepackage{listings}
\usepackage{courier}
%\usepackage[utf8]{inputenc}
%\usepackage[russian]{babel}

\lstset{
  numbers=left,
  basicstyle=\ttfamily\footnotesize,
  numberstyle=\tiny\color{gray},
  stepnumber=1,
  numbersep=10pt,
}

\newcommand{\iu}{\ensuremath{\mathrm{i}}}
\newcommand{\bbR}{\mathbb{R}}
\newcommand{\bbC}{\mathbb{C}}
\newcommand{\calV}{\mathcal{V}}
\newcommand{\calW}{\mathcal{W}}
\newcommand{\macheps}{\epsilon_{\mathrm{mach}}}
\newcommand{\matlab}{\textsc{Matlab}}

\newcommand{\ddiag}{\operatorname{diag}}
\newcommand{\fl}{\operatorname{fl}}
\newcommand{\nnz}{\operatorname{nnz}}
\newcommand{\tr}{\operatorname{tr}}
\renewcommand{\vec}{\operatorname{vec}}

\newcommand{\vertiii}[1]{{\left\vert\kern-0.25ex\left\vert\kern-0.25ex\left\vert #1
    \right\vert\kern-0.25ex\right\vert\kern-0.25ex\right\vert}}
\newcommand{\ip}[2]{\langle #1, #2 \rangle}
\newcommand{\ipx}[2]{\left\langle #1, #2 \right\rangle}
\newcommand{\order}[1]{O( #1 )}

\newcommand{\kron}{\otimes}


\newcommand{\hdr}[2]{
  \pagestyle{fancy}
  \lhead{Bindel, Fall 2019}
  \rhead{Matrix Computation}
  \fancyfoot{}
  \begin{center}
    {\large{\bf HW for #1}} \\ (due: #2)
  \end{center}
  \lstset{language=matlab,columns=flexible}
}


\begin{document}

\hdr{2019-10-18}

% Least squares and minimal norm
% Feature maps and the kernel trick
% Placing parens
% Fast computation (low rank case)

\section{Least squares and minimal norm problems}

% Least squares and minimal norm
% Common perspective via regularization

The least squares problem with Tikhonov regularization is
\[
  \mbox{minimize } \frac{1}{2} \|Ax-b\|_2^2 + \frac{\eta^2}{2} \|x\|^2.
\]
The Tikhonov regularized problem is useful for understanding the
connection between least squares solutions to overdetermined problems
and minimal norm solutions to underdetermined problem.
For $\eta > 0$, the system admits a unique solution independent of whether $m
\geq n$ or $m < n$, and independent of whether or not $A$ has maximal
rank.  The limit when $\eta \rightarrow 0$ is also well-defined:
it is the smallest norm $x$ that minimizes the residual $\|Ax-b\|$.

We usually write the Tikhonov-regularized solution as
\[
  x_\eta = (A^T A + \eta^2 I)^{-1} A^T b.
\]
However, we can get some interesting insights by writing the
regularized normal equations
\[
  \begin{bmatrix} -I & A \\ A^T & \eta^2 I \end{bmatrix}
  \begin{bmatrix} r_\eta \\ x_\eta \end{bmatrix} =
  \begin{bmatrix} b \\ 0 \end{bmatrix}.
\]
The first equation in this system defines the residual ($r = Ax-b$),
while the second gives the regularized version of the normal equation
($A^T r + \eta^2 x = 0$).  Eliminating the $r$ variable gives us the
regularized normal equation in the form we have seen before; but we
can also eliminate $x$ to yield
\[
  (-I - \eta^{-2} A A^T) r_\eta = b.
\]
Scaling variables, we have
\[
  (AA^T + \eta^2 I) r_\eta = -\eta^2 b,
\]
and by substituting into the equation $A^T r + \eta^2 x = 0$, we have
an alternate expression for the solution to the regularized problem:
\[
  x_\eta = A^T (AA^T + \eta^2 I)^{-1} b.
\]

Thus, playing around with the regularized normal equations gives us
two different expressions for $x_\eta$:
\begin{align*}
  x_\eta &= (A^T A + \eta^2 I)^{-1} b \\
        &= A^T (AA^T + \eta^2 I)^{-1} b
\end{align*}
In the full-rank overdetermined case ($m > n$), the former expression
gives us the usual least-squares solutions $(A^T A)^{-1} A^T b$;
in the full-rank under-determined case ($m < n$), the latter
expression gives us the usual minimum-norm solution $A^T (AA^T)^{-1} b$.

For the majority of this lecture, we will focus on the minimum-norm
solution to overdetermined problems and its role in {\em kernel methods}.
However, the connection between the regularized form of the
minimum-norm solution in the overdetermined case and the regularized
form of the least squares problem in the underdetermined case will
be relevant to a discussion at the end of the lecture on
(one) fast method for kernel-based fitting.

\section{Feature maps and the kernel trick}

% Introduce feature maps
% Example of low-degree polynomials
% Problem of singularity (fixed case)
% Higher-dimensional feature maps and minimal norm
% The kernel trick and discarding the feature map

In our first lecture on least squares, we described one of the
standard uses of least squares: fitting a linear model to data.
That is, given (possibly noisy) observations $y_i = f(x_i)$
for $x \in \bbR^N$, we fit $f(x) \approx s(x) = x^T \beta$
by minimizing the squared error:
\[
  \min_\beta \|X\beta - y\|^2,
\]
where $X$ is a matrix whose $i$th row is the vector of coordinates for
the $i$th data point.  Even in simple applications of least squares,
however, a purely linear model may not be adequate for modeling $f$;
we might at least want to consider affine or polynomial functions in
the coordinates, if not something more common.  A simple way to get
more complex models is to introduce a {\em feature map} that takes
our original points in $\bbR^n$ and maps them into a
higher-dimensional space where we will fit our linear models
\[
  s(x) = \phi(x)^T \beta, \quad \phi : \bbR^n \rightarrow \bbR^N.
\]
The {\em features} $\phi_1, \ldots, \phi_N$ are chosen in advance;
the regression coefficients $\beta$ are fit according to the data.
When $m > N$, we would fit the regression coefficients by minimizing
the residual norm $\|\Phi \beta - y\|$, where $[\Phi]_{ij} =  \phi_j(x_i)$.
But often we are interested in the case when $N \gg m$,
in which case we seek a minimal norm solution to the overdetermined
problem, i.e.
\[
  \beta = \Phi^T (\Phi \Phi^T)^{-1} y.
\]
Substituting this into our formula for $s$, we have
\[
  s(x) = \phi(x)^T \Phi^T (\Phi \Phi^T)^{-1} y.
\]

Now, define the {\em kernel function} $k(x, x') = \phi(x)^T \phi(x')$;
then we can rewrite $s(x)$ in terms of the kernel function as
\[
  s(x) = k_{xX} (K_{XX})^{-1} y
\]
where $X = (x_1, x_2, \ldots, x_m)$ is the list of sample coordinates,
and the subscript $X$ means ``form a matrix or vector where $x_1,
\ldots, x_m$ are inserted into this argument in turn,'' i.e.
\begin{align*}
  k_{xX} &= \begin{bmatrix} k(x, x_1) & k(x, x_2) & \ldots & k(x, x_m)
           \end{bmatrix}, \\
  K_{XX} &=
  \begin{bmatrix}
    k(x_1, x_1) & \ldots & k(x_1, x_m) \\
    \vdots & \ddots & \vdots \\
    k(x_m, x_2) & \ldots & k(x_m, x_m)
  \end{bmatrix}.
\end{align*}
Having expressed our interpolant purely in terms of the kernel
function, we can now dispense with the feature map and the
corresponding $\beta$ coefficient: only the kernel matters.
For common kernels used in approximation theory and statistics
(such as the Mat\'ern family or the squared exponential kernels),
we usually don't bother to write down an associated feature map.

\section{Placing parens and alternate interpretations}

% s = a' A inv(A'A) y
%   = a' [A inv(A'A) y)]  -- min norm solution
%   = k_xX inv(K_XX) y    -- kernel trick
%   = k_xX (K_XX \ y)     -- usual kernel formulation + basis fns
%   = (k_xX/K_XX) y       -- data dependent basis / fitting features

The expression
\[
  s(x) = \phi(x)^T \Phi^T (\Phi \Phi^T)^{-1} y
\]
involves a product of several terms that we can group in different
ways:
\begin{align*}
  s(x) &= \phi(x)^T \beta, & \beta = \Phi^\dagger y \\
  s(x) &= k_{xX} c, & c = K_{XX}^{-1} y \\
  s(x) &= d(x)^T y, & d(x) = K_{XX}^{-1} K_{Xx} = (\Phi^T)^\dagger \phi(x)
\end{align*}
We have already discussed the meaning of the first of these groupings,
with $\beta$ as a minimal-norm solution to an overdetermined linear
system relating features to observations.  We now comment on the other
two.

The expression
\[
  s(x) = k_{xX} c = \sum_{i=1}^m k(x,x_i) c_i
\]
involves basis functions $x \mapsto k(x,x_i)$ depending on the
location of the data sites $x_1, \ldots, x_m$.  Many common kernels
depend only on the distance between the two arguments; for example,
the squared exponential kernel is
\[
  k^{\mbox{SE}}(x, x') = \psi(\|x-x'\|; \sigma), \quad
  \psi(r) = \exp(-r^2/2 \sigma^2).
\]
In this case, we would have
\[
  s(x) = \sum_{i=1}^m \psi(\|x-x_i\|) c_i,
\]
i.e.~$s(x)$ is a linear combination of {\em translates} of the function
$\psi$.  The coefficients $c_i$ are simply chosen to satisfy the
interpolation conditions.

The expression
\[
  s(x) = d(x)^T y = \sum_{i=1}^m d_i(x) y_i
\]
is an expansion of $s$ in terms of the {\em Lagrange functions} $d_i$,
which satisfy $d_i(x_j) = \delta_{ij}$.  Another way of thinking about
$d(x)$ involves the least squares formulation:
\[
  \mbox{minimize } \|\Phi^T d(x) - \phi(x)\|^2.
\]
Why is this a sensible thing to do?  The least squares formulation is
attempting to solve the approximation problem
\[
  \phi_i(x) \approx \sum_{j=1}^m \phi_i(x_j) d_j(x)
\]
in a least squares sense; that is, for a collection of representative
functions (the features), we are trying to predict the value at $x$
as a linear combination of values at the sample points $x_1, \ldots, x_m$.
Once we have that combination, together with the function values
$f(x_1), \ldots, f(x_m)$ (the $y$ vector), we use the same linear
combination to predict the value of $f(x)$.

\section{From kernels back to least squares}

% Point out that there is a lot of structure in general!
% Regularized kernel formulation
% Hidden regularized least squares
% Transfer via the residual

While there are several interpretations for the kernel system,
in practice we usually compute
\[
  K_{XX} c = y
\]
and then predict using $s(x) = k_{xX} c$.  In general, this costs
$O(m^3)$ time for the initial fit and $O(m)$ time to evaluate the
interpolant.  However, we can sometimes use the structure of the
kernel to more quickly compute the coefficients or predict at new
points.  Standard approaches typically exploit either {\em smoothness}
of the kernel or {\em low-dimensional} structure of the distribution
of points in the original space.  We will briefly discuss the former,
using the connection between minimal norm problems and least squares
that we discussed earlier in the lecture.

For very smooth kernel functions with long length scales relative
to the spacing between points, the kernel matrix $K_{XX}$ --- though
positive definite --- will be very ill-conditioned.  In this case,
we often work with a regularized version of the fitting problem:
\[
  (K_{XX} + \eta^2 I) c = y
\]
Often far fewer than $m$ eigenvalues of $K_{XX}$ that are much greater
than $\eta^2$, and so we can effectively approximate the system by
\[
  (AA^T + \eta^2 I) \hat{c} = y.
\]
Here, we can think of the rows of $A$ as being ``reduced'' feature
vectors.  From our earlier discussion, we recognize that $\hat{c}$
is the scaled residual for a regularized least squares problem with
$A$; that is, if we solve
\[
  \mbox{minimize } \frac{1}{2} \|Au-b\|^2 + \frac{\eta^2}{2} \|u\|^2
\]
then  
\[
  \hat{c} = \eta^{-2} (b-Au) = -\eta^{-2} r.
\]
Moreover, suppose we know how to compute the reduced feature vector
at an evaluation point $x$, i.e.~we can find $a_x$ such that
\[
  a_x^T A^T = k_{xX}.
\]
Then using the regularized normal equation $A^T r + \eta^2 u = 0$, we have
\begin{align*}
  k_{xX} \hat{c} = a_x^T A^T (-\eta^{-2} r) = a_x^T u.
\end{align*}
That is, solving the regularized kernel problem is (up to error
associated with a low-rank approximation) equivalent to
solving a regularized least squares problem, and we get the same
predictions whether we compute the least squares predictor $a_x^T u$
or the kernel-based predictor $k_{xX} \hat{c}$.

\end{document}

\documentclass[12pt, leqno]{article} %% use to set typesize
\usepackage{fancyhdr}
\usepackage[sort&compress]{natbib}
\usepackage[letterpaper=true,colorlinks=true,linkcolor=black]{hyperref}

\usepackage{amsfonts}
\usepackage{amsmath}
\usepackage{amssymb}
\usepackage{color}
\usepackage{tikz}
\usepackage{pgfplots}
\usepackage{listings}
\usepackage{courier}
%\usepackage[utf8]{inputenc}
%\usepackage[russian]{babel}

\lstset{
  numbers=left,
  basicstyle=\ttfamily\footnotesize,
  numberstyle=\tiny\color{gray},
  stepnumber=1,
  numbersep=10pt,
}

\newcommand{\iu}{\ensuremath{\mathrm{i}}}
\newcommand{\bbR}{\mathbb{R}}
\newcommand{\bbC}{\mathbb{C}}
\newcommand{\calV}{\mathcal{V}}
\newcommand{\calW}{\mathcal{W}}
\newcommand{\macheps}{\epsilon_{\mathrm{mach}}}
\newcommand{\matlab}{\textsc{Matlab}}

\newcommand{\ddiag}{\operatorname{diag}}
\newcommand{\fl}{\operatorname{fl}}
\newcommand{\nnz}{\operatorname{nnz}}
\newcommand{\tr}{\operatorname{tr}}
\renewcommand{\vec}{\operatorname{vec}}

\newcommand{\vertiii}[1]{{\left\vert\kern-0.25ex\left\vert\kern-0.25ex\left\vert #1
    \right\vert\kern-0.25ex\right\vert\kern-0.25ex\right\vert}}
\newcommand{\ip}[2]{\langle #1, #2 \rangle}
\newcommand{\ipx}[2]{\left\langle #1, #2 \right\rangle}
\newcommand{\order}[1]{O( #1 )}

\newcommand{\kron}{\otimes}


\newcommand{\hdr}[2]{
  \pagestyle{fancy}
  \lhead{Bindel, Fall 2019}
  \rhead{Matrix Computation}
  \fancyfoot{}
  \begin{center}
    {\large{\bf HW for #1}} \\ (due: #2)
  \end{center}
  \lstset{language=matlab,columns=flexible}
}


\begin{document}

\hdr{2019-10-11}

\section{A cautionary tale}

You have been dropped on a desert island with a laptop with a magic
battery of infinite life, a MATLAB license, and a complete lack of
knowledge of basic geometry.  In particular, while you know about
least squares fitting, you have forgotten how to compute the perimeter
of a square.  You vaguely feel that it ought to be related to the
perimeter or side length, though, so you set up the following model:
\[
  \mbox{perimeter} = \alpha \cdot \mbox{side length} + \beta \cdot \mbox{diagonal}.
\]
After measuring several squares, you set up a least squares system
$Ax = b$; with your real eyes, you know that this must look
like
\[
  A = \begin{bmatrix} s & \sqrt{2} s \end{bmatrix}, \quad
  b = 4 s
\]
where $s$ is a vector of side lengths.  The normal equations are
therefore
\[
A^T A = \|s\|^2
\begin{bmatrix} 1 & \sqrt{2} \\ \sqrt{2} & 2 \end{bmatrix}, \quad
A^T b = \|s\|^2
\begin{bmatrix} 4 \\ 4 \sqrt{2} \end{bmatrix}.
\]
This system does have a solution; the problem is that it has far more
than one.  The equations are singular, but consistent.
We have no data that would lead us to prefer to write
$p = 4s$ or $p = 2 \sqrt{2} d$ or something in between.
The fitting problem is {\em ill-posed}.

We deliberately started with an extreme case, but some ill-posedness
is common in least squares problems.  As a more natural example,
suppose that we measure the height, waist girth, chest girth, and
weight of a large number of people, and try to use these factors to
predict some other factor such as proclivity to heart disease.  Naive
linear regression -- or any other naively applied statistical
estimation technique -- is likely to run into trouble, as the height,
weight, and girth measurements are highly correlated.  It is not that
we cannot fit a good linear model; rather, we have too many models
that are each almost as good as the others at fitting the data!  We
need a way to choose between these models, and this is the point of
{\em regularization}.

\section{Bias-variance tradeoffs in the matrix setting}

Least squares is often used to fit a model to be used for prediction
in the future.  In learning theory, there is a notion of {\em
  bias-variance} decomposition of the prediction error: the prediction
error consists of a bias term due to using a space of models that does
not actually fit the data, and a term that is related to variance in
the model as a function of measurement noise on the input.  These are
concepts that we can connect concretely to the type of sensitivity
analysis we have seen before, a task we turn to now.

Suppose $A \in \bbR^{M \times n}$ is a matrix of factors that we wish
to use in predicting the entries of $b \in \bbR^M$ via the linear
model
\[
  Ax \approx b.
\]
We partition $A$ and $b$ into the first $m$ rows (where we have
observations) and the remaining $M-m$ rows (where we wish to use
the model for prediction):
\[
  A = \begin{bmatrix} A_1 \\ A_2 \end{bmatrix}, \quad
  b = \begin{bmatrix} b_1 \\ b_e \end{bmatrix}
\]
If we could access all of $b$, we would compute $x$ by the least
square problem
\[
  Ax = b + r, \quad r \perp \mathcal{R}(A).
\]
In practice, we are given only $A_1$ and $b_1 + e$ where
$e$ is a vector of random errors, and we fit the model coefficients
$\hat{x}$ by solving
\[
  \mbox{minimize } \|A_1 \hat{x} - (b_1 + e)\|^2.
\]
Our question, then: what is the least squared error in using $\hat{x}$
for prediction, and how does it compare to the best error possible?
That is, what is the relation between $\|A \hat{x}-b\|^2$ and
$\|r\|^2$?

Note that
\[
  A\hat{x}-b = A(\hat{x}-x) + r
\]
and by the Pythagorean theorem and orthogonality of the residual,
\[
  \|A\hat{x}-b\|^2 = \|A(\hat{x}-x)\|^2 + \|r\|^2.
\]
The term $\|\hat{r}\|^2$ is the (squared) bias term, the part of the
error that is due to lack of power in our model.  The term
$\|A(\hat{x}-x)\|^2$ is the variance term, and is associated with
sensitivity of the fitting process.  If we dig further into this,
we can see that
\begin{align*}
  x &= A_1^\dagger (b_1 + r_1) &
  \hat x &= A_1^\dagger (b_1 + e),
\end{align*}
and so
\[
  \|A(\hat x - x)\|^2 = \|A A_1^\dagger (e-r_1)\|^2
\]
Taking norm bounds, we find
\[
  \|A(\hat x - x)\| \leq \|A\| \|A_1^\dagger\| (\|e\| + \|r_1)\|),
\]
and putting everything together,
\[
  \|A\hat{x}-b\| \leq (1+\|A\| \|A_1^\dagger\|) \|r\| + \|A\|
  \|A_1^\dagger\| \|e\|.
\]
If there were no measurement error $e$, we would have a
{\em quasi-optimality} bound saying that the squared error in prediction
via $\hat{x}$ is within a factor of $1 + \|A\| \|A_1^\dagger\|$ of the best
squared error available for any similar model.  If we scale the factor
matrix $A$ so that $\|A\|$ is moderate in size, everything boils down
to $\|A_1^\dagger\|$.

When $\|A_1^\dagger\|$ is large, the problem of fitting to training
data is ill-posed, and the accuracy can be compromised.  What can we
do?  As we discussed in the last section, the problem with ill-posed
problems is that they admit many solutions of very similar quality.
In order to distinguish between these possible solutions to find a
model with good predictive power, we consider {\em regularization}:
that is, we assume that the coefficient vector $x$ is not too large
in norm, or that it is sparse.  Different statistical assumptions give
rise to different regularization strategies; for the current
discussion, we shall focus on the computational properties
of a few of the more common regularization strategies without going
into the details of the statistical assumptions.  In particular,
we consider four strategies in turn
\begin{enumerate}
\item {\em Factor selection} via {\em pivoted QR}.
\item {\em Tikhonov regularization} and its solution.
\item {\em Truncated SVD regularization}.
\item {\em $\ell^1$ regularization} or the {\em lasso}.
\end{enumerate}

\section{Factor selection and pivoted QR}

In ill-conditioned problems, the columns of $A$ are nearly linearly
dependent; we can effectively predict some columns as linear
combinations of other columns.  The goal of the column pivoted QR
algorithm is to find a set of columns that are ``as linearly
independent as possible.''  This is not such a simple task,
and so we settle for a greedy strategy: at each step, we select the
column that is least well predicted (in the sense of residual norm)
by columns already selected.  This leads to the {\em pivoted QR
  factorization}
\[
  A \Pi = Q R
\]
where $\Pi$ is a permutation and the diagonal entries of $R$ appear
in descending order (i.e. $r_{11} \geq r_{22} \geq \ldots$).  To
decide on how many factors to keep in the factorization, we either
automatically take the first $k$ or we dynamically choose to take $k$
factors where $r_{kk}$ is greater than some tolerance and
$r_{k+1,k+1}$ is not.

The pivoted QR approach has a few advantages.  It yields {\em
  parsimonious} models that predict from a subset of the columns of
$A$ -- that is, we need to measure fewer than $n$ factors to produce
an entry of $b$ in a new column.  It can also be computed relatively
cheaply, even for large matrices that may be sparse.
However, pivoted QR is not the only approach!  A related approach
due to Golub, Klema, and Stewart computes $A = U \Sigma V^T$ and
chooses a subset of the factors based on pivoted QR of $V^T$.
More generally, approaches such as the lasso yield an automatic
factor selection.

\section{Truncated SVD}

The Tikhonov filter reduces the effect of small singular values on
the solution, but it does not eliminate that effect.  By contrast,
the {\em truncated SVD} approach uses the filter
\[
f(z) =
\begin{cases}
  z, & z > \sigma_{\min} \\
  \infty, & \mbox{otherwise}.
\end{cases}
\]
In other words, in the truncated SVD approach, we use
\[
  x = V_k \Sigma_k^{-1} U_k^T b
\]
where $U_k$ and $V_k$ represent the leading $k$ columns of $U$ and
$V$, respectively, while $\Sigma_k$ is the diagonal matrix consisting
of the $k$ largest singular values.

\section{$\ell^1$ and the lasso}

An alternative to Tikhonov regularization (based on a Euclidean norm
of the coefficient vector) is an $\ell^1$ regularized problem
\[
  \mbox{minimize } \|Ax-b\|^2 + \lambda \|x\|_1.
\]
This is sometimes known as the ``lasso'' approach.  The $\ell^1$
regularized problem has the property that the solutions tend to
become sparse as $\lambda$ becomes larger.  That is, the $\ell^1$
regularization effectively imposes a factor selection process like
that we saw in the pivoted QR approach.  Unlike the pivoted QR
approach, however, the $\ell^1$ regularized solution cannot be
computed by one of the standard factorizations of numerical linear
algebra.  Instead, one treats it as a more general {\em convex
  optimization} problem.  We will discuss some approaches to the
solution of such problems later in the semester.

\section{Tradeoffs and tactics}

All four of the regularization approaches we have described are used
in practice, and each has something to recommend it.  The pivoted QR
approach is relatively inexpensive, and it results in a model that
depends on only a few factors.  If taking the measurements to compute
a prediction costs money --- or even costs storage or bandwidth for
the factor data! --- such a model may be to our advantage.  The
Tikhonov approach is likewise inexpensive, and has a nice Bayesian
interpretation (though we didn't talk about it).  The truncated SVD
approach involves the best approximation rank $k$ approximation to the
original factor matrix, and can be interpreted as finding the $k$ best
factors that are linear combinations of the original measurements.
The $\ell_1$ approach again produces models with sparse coefficients;
but unlike QR with column pivoting, the $\ell_1$ regularized solutions
incorporate information about the vector $b$ along with the matrix $A$.

So which regularization approach should one use?  In terms of
prediction quality, all can provide a reasonable deterrent against
ill-posedness and overfitting due to highly correlated factors.  Also,
all of the methods described have a parameter (the number of retained
factors, or a penalty parameter $\lambda$) that governs the tradeoff
between how well-conditioned the fitting problem will be and the
increase in bias that naturally comes from looking at a smaller class
of models.  Choosing this tradeoff intelligently may be rather more
important than the specific choice of regularization strategy.  A
detailed discussion of how to make this tradeoff is beyond the scope
of the class; but we will see some of the computational tricks
involved in implementing specific strategies for choosing
regularization parameters before we are done.

\end{document}

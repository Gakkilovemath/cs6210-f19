\documentclass[12pt, leqno]{article} %% use to set typesize
\usepackage{fancyhdr}
\usepackage[sort&compress]{natbib}
\usepackage[letterpaper=true,colorlinks=true,linkcolor=black]{hyperref}

\usepackage{amsfonts}
\usepackage{amsmath}
\usepackage{amssymb}
\usepackage{color}
\usepackage{tikz}
\usepackage{pgfplots}
\usepackage{listings}
\usepackage{courier}
%\usepackage[utf8]{inputenc}
%\usepackage[russian]{babel}

\lstset{
  numbers=left,
  basicstyle=\ttfamily\footnotesize,
  numberstyle=\tiny\color{gray},
  stepnumber=1,
  numbersep=10pt,
}

\newcommand{\iu}{\ensuremath{\mathrm{i}}}
\newcommand{\bbR}{\mathbb{R}}
\newcommand{\bbC}{\mathbb{C}}
\newcommand{\calV}{\mathcal{V}}
\newcommand{\calW}{\mathcal{W}}
\newcommand{\macheps}{\epsilon_{\mathrm{mach}}}
\newcommand{\matlab}{\textsc{Matlab}}

\newcommand{\ddiag}{\operatorname{diag}}
\newcommand{\fl}{\operatorname{fl}}
\newcommand{\nnz}{\operatorname{nnz}}
\newcommand{\tr}{\operatorname{tr}}
\renewcommand{\vec}{\operatorname{vec}}

\newcommand{\vertiii}[1]{{\left\vert\kern-0.25ex\left\vert\kern-0.25ex\left\vert #1
    \right\vert\kern-0.25ex\right\vert\kern-0.25ex\right\vert}}
\newcommand{\ip}[2]{\langle #1, #2 \rangle}
\newcommand{\ipx}[2]{\left\langle #1, #2 \right\rangle}
\newcommand{\order}[1]{O( #1 )}

\newcommand{\kron}{\otimes}


\newcommand{\hdr}[2]{
  \pagestyle{fancy}
  \lhead{Bindel, Fall 2019}
  \rhead{Matrix Computation}
  \fancyfoot{}
  \begin{center}
    {\large{\bf HW for #1}} \\ (due: #2)
  \end{center}
  \lstset{language=matlab,columns=flexible}
}


\begin{document}

\hdr{HW 1}{2019-09-09}

You may (and should) talk about problems with each other and with me,
providing attribution for any good ideas you might get.  Your final
write-up should be your own.

\paragraph*{1: About you}
Tell me a few things about yourself:
\begin{itemize}
\item How do you prefer to be called?
\item Why are you taking the class?
\item Are there things you particularly hope to see?
\item Do you have any concerns (about background, schedule, etc)?
\end{itemize}

\paragraph*{2: A problem of performance}
Julia and \matlab\ support both sparse and dense matrix data structures,
and they have different performance characteristics.  For a variety
of square matrices of size $n$ and sparsity level $s$ (where $s$ is
the fraction of entries that are nonzero), compare the speed of dense
and sparse matrix-vector multiply.  You may use {\tt As = sparse(A)}
to make a sparse version of a dense matrix $A$.  What do you observe
about the relative performance of these options?

{\em Note:} If you want examples of how to write timing tests, Julia
scripts will be added to the \matlab\ scripts already in the class
repository.

\paragraph*{3: Seeking structure}
Rewrite the following code fragment to run in $O(n)$ time
(in \matlab; Julia code and tests at \url{https://github.com/dbindel/cs6210-f19/tree/master/hw/code}).
\begin{lstlisting}
  % u, v, and x are length n
  A = eye(n) + u*v';
  
  y  = A*x;
  z  = A'*x;
  d  = diag(A);
  df = diag(flipud(A));
  t  = trace(A);
  c  = det(A);
\end{lstlisting}

{\em Hint (last line):} For any $X, Y \in \bbR^{n \times k}$:
$\det(I+XY^T) = \det(I+Y^T X)$.

\end{document}

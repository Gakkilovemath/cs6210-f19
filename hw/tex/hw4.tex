\documentclass[12pt, leqno]{article} %% use to set typesize
\usepackage{fancyhdr}
\usepackage[sort&compress]{natbib}
\usepackage[letterpaper=true,colorlinks=true,linkcolor=black]{hyperref}

\usepackage{amsfonts}
\usepackage{amsmath}
\usepackage{amssymb}
\usepackage{color}
\usepackage{tikz}
\usepackage{pgfplots}
\usepackage{listings}
\usepackage{courier}
%\usepackage[utf8]{inputenc}
%\usepackage[russian]{babel}

\lstset{
  numbers=left,
  basicstyle=\ttfamily\footnotesize,
  numberstyle=\tiny\color{gray},
  stepnumber=1,
  numbersep=10pt,
}

\newcommand{\iu}{\ensuremath{\mathrm{i}}}
\newcommand{\bbR}{\mathbb{R}}
\newcommand{\bbC}{\mathbb{C}}
\newcommand{\calV}{\mathcal{V}}
\newcommand{\calW}{\mathcal{W}}
\newcommand{\macheps}{\epsilon_{\mathrm{mach}}}
\newcommand{\matlab}{\textsc{Matlab}}

\newcommand{\ddiag}{\operatorname{diag}}
\newcommand{\fl}{\operatorname{fl}}
\newcommand{\nnz}{\operatorname{nnz}}
\newcommand{\tr}{\operatorname{tr}}
\renewcommand{\vec}{\operatorname{vec}}

\newcommand{\vertiii}[1]{{\left\vert\kern-0.25ex\left\vert\kern-0.25ex\left\vert #1
    \right\vert\kern-0.25ex\right\vert\kern-0.25ex\right\vert}}
\newcommand{\ip}[2]{\langle #1, #2 \rangle}
\newcommand{\ipx}[2]{\left\langle #1, #2 \right\rangle}
\newcommand{\order}[1]{O( #1 )}

\newcommand{\kron}{\otimes}


\newcommand{\hdr}[2]{
  \pagestyle{fancy}
  \lhead{Bindel, Fall 2019}
  \rhead{Matrix Computation}
  \fancyfoot{}
  \begin{center}
    {\large{\bf HW for #1}} \\ (due: #2)
  \end{center}
  \lstset{language=matlab,columns=flexible}
}


\begin{document}

\hdr{2019-09-23}{2019-10-02}

You may (and should) talk about problems with each other and with me,
providing attribution for any good ideas you might get.  Your final
write-up should be your own.

% GECP and function approximation
% Left-looking vs right-looking
% Bordered system solver (already did this a little)
% Convergence of iterative refinement "stop making progress at some point"

\paragraph*{1: GECP and function approximation}
Suppose for a smooth function of two variables on the domain
$[-1,1]^2$, we seek an approximation as a sum of {\em separable functions}:
\[
  f_m(x,y) = \sum_{k=1}^m g_k(x) h_k(y).
\]
One approach (due to Townsend and Trefethen) is the following greedy
strategy: given $f_m(x,y)$ with $f_0 \equiv 0$, compute $f_{m+1}$ as
\begin{align*}
  e_{m}(x,y) &= f(x,y) - f_m(x,y) \\
  x_m^*, y_m^* &= \operatorname{argmax}_{x,y} |e_m(x,y)| \\
  g_{m+1}(x) &= e_m(x, y_m^*) / e_m(x_m^*, y_m^*) \\
  h_{m+1}(y) &= e_m(x_m^*, y).
\end{align*}
Townsend and Trefethen refer to this as an infinite-dimensional
version of Gaussian elimination with complete pivoting.  Explain why
this is a reasonable description, and implement the strategy yourself
(you may restrict your attention to a finite mesh in $x$ and $y$, if
you wish).  Illustrate the convergence of the maximum error over
$[-1,1]^2$ in the first twenty terms of approximating the squared
exponential function $f(x,y) = \exp(-(x^2+y^2)/2)$ and the Ackley function
\[
  f(x,y) =
  20
  -20 \exp\left( -0.2 \sqrt{\frac{x^2+y^2}{2}} \right)
  -\exp\left( \frac{1}{2} \cos(2\pi x) + \frac{1}{2} \cos(2\pi y) \right).
\]
You may use the partial code templates provided
\href{https://github.com/dbindel/cs6210-f19/tree/master/hw/code/hw4}{in the Github
  repo}
(in Julia or \matlab), or you may write your own.  Note that you will
most likely want some mechanism to allow early termination if the
error maximum of $|e_m|$ becomes too small (either in an absolute
sense or relative to the maximum of $|f|$).

\paragraph*{2: Convergence of iterative refinement}
In the second section of the lecture notes for Sep 23, we gave an
informal discussion of the convergence of iterative refinement.  In
this problem, we nail down the analysis a little more.  Consider the
refinement iteration
\[
  x_{k+1} = x_k - \hat{A}_k^{-1} (b - A x_k + g_k) + h_k
\]
where $\hat{A}_k = A + E_k$ and $g_k$ and $h_k$ denote the rounding
errors associated with computing the residual $b-Ax_k$ and doing the
final subtraction in floating point arithmetic.  Assuming $\|E_k\|
\leq \eta$, $\|g_k\| \leq \gamma$, and $\|h_k\| \leq \nu$ for all $k$,
and assuming that $\|A^{-1}\| < \eta^{-1}/2$, use a geometric series to
bound the error at step $k$ in terms of the initial error and the
rounding errors introduced at each step.  Because of the presence of
$g_k$ and $h_k$, the error will {\em not} converge to zero; what is an
asymptotic bound on the norm of the error as $k \rightarrow \infty$?
What does your analysis suggest about the number of steps of iterative
refinement that one should take?

{\em Hint:}  Observe that in general if one has non-negative
quantities $\beta_k$ satisfying
\[
  \beta_{k+1} \leq \rho \beta_k + \delta
\]
then by recurrence
\[
  \beta_k
  \leq \rho^k \beta_0 + \frac{1-\rho^k}{1-\rho} \delta,
\]
and by a geometric series bound, for $0 \leq \rho < 1$,
\[
  \beta_k \leq \rho^k \beta_0 + \frac{\delta}{1-\rho}
\]

\paragraph*{3: Condititioning of the complement}
Suppose $A$ is symmetric and positive definite.  Show that if $S$ is a
Schur complement that appears during (unpivoted) factorization of $A$, then
$\kappa_2(S) \leq \kappa_2(A)$.  Show by example that this statement
may not hold if $A$ is not positive definite.

\end{document}

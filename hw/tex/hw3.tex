\documentclass[12pt, leqno]{article} %% use to set typesize
\usepackage{fancyhdr}
\usepackage[sort&compress]{natbib}
\usepackage[letterpaper=true,colorlinks=true,linkcolor=black]{hyperref}

\usepackage{amsfonts}
\usepackage{amsmath}
\usepackage{amssymb}
\usepackage{color}
\usepackage{tikz}
\usepackage{pgfplots}
\usepackage{listings}
\usepackage{courier}
%\usepackage[utf8]{inputenc}
%\usepackage[russian]{babel}

\lstset{
  numbers=left,
  basicstyle=\ttfamily\footnotesize,
  numberstyle=\tiny\color{gray},
  stepnumber=1,
  numbersep=10pt,
}

\newcommand{\iu}{\ensuremath{\mathrm{i}}}
\newcommand{\bbR}{\mathbb{R}}
\newcommand{\bbC}{\mathbb{C}}
\newcommand{\calV}{\mathcal{V}}
\newcommand{\calW}{\mathcal{W}}
\newcommand{\macheps}{\epsilon_{\mathrm{mach}}}
\newcommand{\matlab}{\textsc{Matlab}}

\newcommand{\ddiag}{\operatorname{diag}}
\newcommand{\fl}{\operatorname{fl}}
\newcommand{\nnz}{\operatorname{nnz}}
\newcommand{\tr}{\operatorname{tr}}
\renewcommand{\vec}{\operatorname{vec}}

\newcommand{\vertiii}[1]{{\left\vert\kern-0.25ex\left\vert\kern-0.25ex\left\vert #1
    \right\vert\kern-0.25ex\right\vert\kern-0.25ex\right\vert}}
\newcommand{\ip}[2]{\langle #1, #2 \rangle}
\newcommand{\ipx}[2]{\left\langle #1, #2 \right\rangle}
\newcommand{\order}[1]{O( #1 )}

\newcommand{\kron}{\otimes}


\newcommand{\hdr}[2]{
  \pagestyle{fancy}
  \lhead{Bindel, Fall 2019}
  \rhead{Matrix Computation}
  \fancyfoot{}
  \begin{center}
    {\large{\bf HW for #1}} \\ (due: #2)
  \end{center}
  \lstset{language=matlab,columns=flexible}
}


\begin{document}

\hdr{2019-09-16}{2019-09-23}

You may (and should) talk about problems with each other and with me,
providing attribution for any good ideas you might get.  Your final
write-up should be your own.

\paragraph*{1: Gauss transformations}
Let $G = I-\tau e_k^T$ be a Gauss transformation matrix (so the only
nonzeros in $\tau$ appear after entry $k$).  Then
\begin{itemize}
\item Show that $G^{-1} = I+\tau e_k^T$.
\item Argue that $\|G\|_\infty = 1 + \|\tau\|_\infty$ and
  $\|G\|_1 = 1 + \|\tau\|_1$.
\item The singular values of $G$ are all one, except for two of them
  that are the positive roots of the equation
  \[
    p(\sigma^2) = 1 - (2+\|\tau\|^2) \sigma^2 + \sigma^4 = 0.
  \]
  Using this fact (which you are not required to prove), write a code
  $\kappa_2(G) = \|G\|_2 \|G^{-1}\|_2 = \sigma_1(G) / \sigma_n(G)$.
  Your code should give remain accurate (to within a few ulps) when
  $\|\tau\|$ is very large or small (including $\tau = 0$).
\end{itemize}
Bonus: Prove the fact about the singular values used in the second part.

\paragraph*{2: Follow the arrow}
Consider the {\em arrow matrix}
\[
  A = \begin{bmatrix} D & b \\ c^T & f \end{bmatrix}
\]
where $D$ is diagonal.
\begin{itemize}
\item Show that $A$ is invertible if the diagonal entries $d_i$ are
  all nonzero and $f-\sum_i b_i c_i / d_i \neq 0$.
\item Write an $O(n)$ time routine to compute $\det(A)$
\item Write an $O(n)$ time routine to solve the system $Ax = y$.
\end{itemize}
Your codes should follow either the \matlab\ or the Julia template in
the class repository.

\end{document}
